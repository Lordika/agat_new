\newcommand\LinearOneA{%
	\ensuremath{% 
		f_1 = (\thedetaa e_1) + (\thedetab e_2);\ 
		f_2 = (\thedetac e_1) + (\thedetad e_2); 
	}
	$$
		A = \begin{pmatrix}
			\thedetba & \thedetbb \\
			\thedetbc & \thedetbd 
		\end{pmatrix}.
	$$%
}

\newcommand\MultiLinearOneA[2][\time]{%
	\reinitrand[seed=#1, first=-5, last=5]%
	\ToNull{tmpa}%
	\whiledo{\value{tmpa}<#2}{%
		\Inc{tmpa}\Inc{primcount}\InitDet[4]\theprimcount\ \LinearOneA \BigPar%\qquad%
	 }%
}

\newcommand\LinearOneB{%
	\ensuremath{% 
		f_1 = (\thedetaa e_1) + (\thedetab e_2) + (\thedetac e_3);\ 
		f_2 = (\thedetad e_1) + (\thedetae e_2) + (\thedetba e_3);\\ 
		f_3 = (\thedetbb e_1) + (\thedetbc e_2) + (\thedetbd e_3);
	}%
	$$
		A = \begin{pmatrix}
			\thedetca & \thedetcb & \thedetcc \\
			\thedetcd & \thedetce & \thedetda \\
			\thedetdb & \thedetdc & \thedetdd
		\end{pmatrix}.
	$$%
}

\newcommand\MultiLinearOneB[2][\time]{%
	\reinitrand[seed=#1, first=-5, last=5]%
	\ToNull{tmpa}%
	\whiledo{\value{tmpa}<#2}{%
		\Inc{tmpa}\Inc{primcount}\InitDet[5]\theprimcount\ \LinearOneB \BigPar%\qquad%
	 }%
}

\newcommand\LinearTwoA{%
	\ensuremath{% 
		a_1 = (\thedetaa, \thedetab);\ 
		a_2 = (\thedetac, \thedetad); 
	}%
	$$
		A = \begin{pmatrix}
			\thedetba & \thedetbb\\
			\thedetbc & \thedetbd
		\end{pmatrix}.
	$$%
	\ensuremath{% 
		b_1 = (\thedetca, \thedetcb);\ 
		b_2 = (\thedetcd, \thedetdd); 
	}%
}

\newcommand\MultiLinearTwoA[2][\time]{%
	\reinitrand[seed=#1, first=-5, last=5]%
	\ToNull{tmpa}%
	\whiledo{\value{tmpa}<#2}{%
		\Inc{tmpa}\Inc{primcount}\InitDet[4]\theprimcount\ \LinearTwoA\BigPar%\qquad%
	 }%
}

\newcommand\LinearTwoB{%
	\ensuremath{% 
		a_1 = (\thedetaa, \thedetab, \thedetac);\ 
		a_2 = (\thedetad, \thedetae, \thedetaf);\ 
		a_3 = (\thedetba, \thedetbb, \thedetbc);
	}%
	$$
		A = \begin{pmatrix}
			\thedetbd & \thedetbe & \thedetbf \\
			\thedetca & \thedetcb & \thedetcc \\
			\thedetcd & \thedetce & \thedetcf
		\end{pmatrix}.
	$$%
	\ensuremath{% 
		b_1 = (\thedetda, \thedetdb, \thedetdc);\ 
		b_2 = (\thedetdd, \thedetde, \thedetdf);\ 
		b_3 = (\thedetea, \thedeteb, \thedetec);
	}%
}

\newcommand\MultiLinearTwoB[2][\time]{%
	\reinitrand[seed=#1, first=-5, last=5]%
	\ToNull{tmpa}%
	\whiledo{\value{tmpa}<#2}{%
		\Inc{tmpa}\Inc{primcount}\InitDet[6]\theprimcount\ \LinearTwoB\BigPar%\qquad%
	 }%
}

\newcommand\LinearThreeA{%
	\ensuremath{% 
		a_1 = (\thedetaa, \thedetab);\ 
		a_2 = (\thedetac, \thedetad);
	}\\%
	\ensuremath{% 
		b_1 = (\thedetba, \thedetbb);\ 
		b_2 = (\thedetbc, \thedetbd);
	}%
}

\newcommand\MultiLinearThreeA[2][\time]{%
	\reinitrand[seed=#1, first=-5, last=5]%
	\ToNull{tmpa}%
	\whiledo{\value{tmpa}<#2}{%
		\Inc{tmpa}\Inc{primcount}\InitDet[4]\theprimcount\ \LinearThreeA\BigPar%\qquad%
	 }%
}

\newcommand\LinearThreeB{%
	\ensuremath{% 
		a_1 = (\thedetaa, \thedetab, \thedetac);\ 
		a_2 = (\thedetad, \thedetae, \thedetaf);\ 
		a_3 = (\thedetba, \thedetbb, \thedetbc);
	}\\ %
	\ensuremath{% 
		b_1 = (\thedetbd, \thedetbe, \thedetbf);\ 
		b_2 = (\thedetca, \thedetcb, \thedetcc);\ 
		b_3 = (\thedetcd, \thedetce, \thedetcf);
	}%
}

\newcommand\MultiLinearThreeB[2][\time]{%
	\reinitrand[seed=#1, first=-5, last=5]%
	\ToNull{tmpa}%
	\whiledo{\value{tmpa}<#2}{%
		\Inc{tmpa}\Inc{primcount}\InitDet[6]\theprimcount\ \LinearThreeB\BigPar%\qquad%
	 }%
}

\newcommand\LinearFourA{%
	$$%
		A = \begin{pmatrix}
			\thedetaa & \thedetab & \thedetac \\
			\thedetba & \thedetbb & \thedetbc \\
			\thedetca & \thedetcb & \thedetcc 
		\end{pmatrix}.
	$$%
}

\newcommand\MultiLinearFourA[2][\time]{%
	\reinitrand[seed=#1, first=-5, last=5]%
	\ToNull{tmpa}%
	\whiledo{\value{tmpa}<#2}{%
		\Inc{tmpa}\Inc{primcount}\InitDet\theprimcount\ \LinearFourA\BigPar%\qquad%
	 }%
}

\newcommand\LinearFourB{%
	$$
		A = \begin{pmatrix}
			\thedetaa & \thedetab & \thedetac & \thedetad\\
			\thedetba & \thedetbb & \thedetbc & \thedetbd\\
			\thedetca & \thedetcb & \thedetcc & \thedetcd\\
			\thedetda & \thedetdb & \thedetdc & \thedetdd
		\end{pmatrix}.
	$$%
}

\newcommand\MultiLinearFourB[2][\time]{%
	\reinitrand[seed=#1, first=-5, last=5]%
	\ToNull{tmpa}%
	\whiledo{\value{tmpa}<#2}{%
		\Inc{tmpa}\Inc{primcount}\InitDet[4]\theprimcount\ \LinearFourB\BigPar%\qquad%
	 }%
}

\newcommand\LinearFiveA{%
	$$
		A = \begin{pmatrix}
			\thedetaa & \thedetab & \thedetac & \thedetad\\
			\thedetba & \thedetbb & \thedetbc & \thedetbd\\
			\thedetca & \thedetcb & \thedetcc & \thedetcd
		\end{pmatrix}.
	$$%
}

\newcommand\MultiLinearFiveA[2][\time]{%
	\reinitrand[seed=#1, first=-5, last=5]%
	\ToNull{tmpa}%
	\whiledo{\value{tmpa}<#2}{%
		\Inc{tmpa}\Inc{primcount}\InitDet[4]\theprimcount\ \LinearFiveA\BigPar%\qquad%
	 }%
}

\newcommand\LinearFiveB{%
	$$
		A = \begin{pmatrix}
			\thedetaa & \thedetab & \thedetac & \thedetad & \thedetae\\
			\thedetba & \thedetbb & \thedetbc & \thedetbd & \thedetab\\
			\thedetca & \thedetcb & \thedetcc & \thedetcd & \thedetac\\
			\thedetda & \thedetdb & \thedetdc & \thedetdd & \thedetad\\
			\thedetea & \thedeteb & \thedetec & \thedeted & \thedetee
		\end{pmatrix}.
	$$%
}

\newcommand\MultiLinearFiveB[2][\time]{%
	\reinitrand[seed=#1, first=-5, last=5]%
	\ToNull{tmpa}%
	\whiledo{\value{tmpa}<#2}{%
		\Inc{tmpa}\Inc{primcount}\InitDet[5]\theprimcount\ \LinearFiveB\BigPar%\qquad%
	 }%
}

\newcommand\LinearSixA{%
	\ensuremath{%
		a = (\thedetaa, \thedetba, \thedetca); \qquad%
		b = (\thedetab, \thedetbb, \thedetcb).% 
	}%
}

\newcommand\MultiLinearSixA[2][\time]{%
	\reinitrand[seed=#1, first=-5, last=5]%
	\ToNull{tmpa}%
	\whiledo{\value{tmpa}<#2}{%
		\Inc{tmpa}\Inc{primcount}\InitDet\theprimcount\ \LinearSixA\BigPar%\qquad%
	 }%
}

\newcommand\LinearSixB{%
	\ensuremath{%
		a = (\thedetaa, \thedetab, \thedetac, \thedetad, \thedetae); \qquad%
		b = (\thedetba, \thedetbb, \thedetbc, \thedetbd, \thedetbe).% 
	}%
}

\newcommand\MultiLinearSixB[2][\time]{%
	\reinitrand[seed=#1, first=-5, last=5]%
	\ToNull{tmpa}%
	\whiledo{\value{tmpa}<#2}{%
		\Inc{tmpa}\Inc{primcount}\InitDet[5]\theprimcount\ \LinearSixB\BigPar%\qquad%
	 }%
}

\newcommand\LinearSevenA{%
	\ensuremath{%
		a_1 = (\thedetaa, \thedetab, \thedetac); \qquad%
		a_2 = (\thedetba, \thedetbb, \thedetbc).% 
	}%
}

\newcommand\MultiLinearSevenA[2][\time]{%
	\reinitrand[seed=#1, first=-5, last=5]%
	\ToNull{tmpa}%
	\whiledo{\value{tmpa}<#2}{%
		\Inc{tmpa}\Inc{primcount}\InitDet\theprimcount\ \LinearSevenA\BigPar%\qquad%
	 }%
}

\newcommand\LinearSevenB{%
	\ensuremath{%
		a_1 = (\thedetaa, \thedetab, \thedetac, \thedetad, \thedetae); \qquad%
		a_2 = (\thedetba, \thedetbb, \thedetbc, \thedetbd, \thedetbe);% 
	}%
}

\newcommand\MultiLinearSevenB[2][\time]{%
	\reinitrand[seed=#1, first=-5, last=5]%
	\ToNull{tmpa}%
	\whiledo{\value{tmpa}<#2}{%
		\Inc{tmpa}\Inc{primcount}\InitDet[5]\theprimcount\ \LinearSevenB\BigPar%\qquad%
	 }%
}

\newcommand\LinearEight{%
	\ensuremath{% 
		f_1 = (\thedetaa, \thedetab, \thedetac)^T;\ % 
		f_2 = (\thedetad, \thedetae, \thedetaf)^T;\ % 
		f_3 = (\thedetba, \thedetbb, \thedetbc)^T;%
	}%
	$$%
		A = \begin{pmatrix}%
			\thedetbd & \thedetbe & \thedetbf \\%
			\thedetca & \thedetcb & \thedetcc \\%
			\thedetcd & \thedetce & \thedetcf%
		\end{pmatrix}; \qquad%
		B = \begin{pmatrix}%
			\thedetda & \thedetdb & \thedetdc \\%
			\thedetdd & \thedetde & \thedetdf \\%
			\thedetea & \thedeteb & \thedetec%
		\end{pmatrix}; \qquad% 
		x = \begin{pmatrix}%
			\thedeted \\ \thedetee \\ \thedetef%
		\end{pmatrix}.%
	$$%
}

\newcommand\MultiLinearEight[1][\time]{%
	\reinitrand[seed=#1, first=-5, last=5]%
	\ToNull{tmpa}%
	\Inc{tmpa} \textbf{\alph{tmpa}). }%
	Проверить, что $f = (f_1; f_2; f_3)$ базис в $R^3$ и найти матрицы перехода $Ce-f$ , $Cf-e$.
	\smallskip \par
	\Inc{tmpa} \textbf{\alph{tmpa}). }%
	Определите координаты вектора $y = A \circ B(x)$ в базисе $f$.
	\smallskip \par
	\Inc{tmpa} \textbf{\alph{tmpa}). }%
	Найти матрицы оператора $A^{-1}$ в базисах $e$ и $f$.
	\smallskip \par
	\Inc{tmpa} \textbf{\alph{tmpa}). }%
	Найти размерность ядра и образа оператора $A$.
	\smallskip \par
	\Inc{tmpa} \textbf{\alph{tmpa}). }%
	Найти размерность ядра и образа оператора $B$.
	\smallskip \par
	\Inc{tmpa} \textbf{\alph{tmpa}). }%
	Постройте ортонормированный базис ядра и образа оператора $A$.
	\smallskip \par
	\Inc{tmpa} \textbf{\alph{tmpa}). }%
	Постройте ортонормированный базис ядра и образа оператора $B$.
	\smallskip \par
	\Inc{tmpa} \textbf{\alph{tmpa}). }%
	Найдите собственные числа и собственные вектора операторов $A$.
	\smallskip \par
	\Inc{tmpa} \textbf{\alph{tmpa}). }%
	Найдите собственные числа и собственные вектора операторов $B$.
	\smallskip \par
	\Inc{tmpa} \textbf{\alph{tmpa}). }%
	Выпишите матрицы операторов $A$ и $B$ в базисе из собственных векторов.
	\smallskip \par
	\Inc{primcount}\InitDet[6]\theprimcount\ \LinearEight\BigPar%\qquad%
	\Inc{primcount}\InitDet[6]\theprimcount\ \LinearEight\BigPar%\qquad%
}
