\Task{Линейный оператор имеет матрицу $A$. Найти его матрицу в базисе $f_1$, $f_2$.}%
{linear_ex01}%

\Task{Линейный оператор имеет матрицу $A$. Найти его матрицу в базисе $f_1$, $f_2$, $f_3$.}%
{linear_ex02}%

\Task{Линейный оператор $f$ в базисе $a_1$, $a_2$ имеет матрицу $A$.  Найти его матрицу в базисе $b_1$, $b_2$.}%
{linear_ex03}%

\Task{Линейный оператор $f$ в базисе $a_1$, $a_2$, $a_3$ имеет матрицу $A$.  Найти его матрицу в базисе $b_1$, $b_2$, $b_3$.}%
{linear_ex04}%

\Task{Доказать, что существует единственное линейное преобразование двумерного пространства, переводящее векторы $a_1$ и $a_2$ в векторы $b_1$ и $b_2$ соответственно. Найти матрицу этого преобразования в том же базисе, в котором заданы координаты векторов.}%
{linear_ex05}%

\Task{Доказать, что существует единственное линейное преобразование трёхмерного пространства, переводящее векторы $a_1$, $a_2$ и $a_3$ в векторы $b_1$, $b_2$ и $b_3$ соответственно. Найти матрицу этого преобразования в том же базисе, в котором заданы координаты векторов.}%
{linear_ex06}%

\Task{Найти базис и размерность ядра и образ линейного оператора $f$, заданного своей матрицей в некотором базисе.}%
{linear_ex07}%

\Task{Найти базис и размерность ядра и образ линейного оператора $f$, заданного своей матрицей в некотором базисе.}%
{linear_ex08}%

\Task{Найти базис и размерность ядра и образ линейного оператора $f$, заданного своей матрицей в некотором базисе.}%
{linear_ex09}%

\Task{Найти базис и размерность ядра и образ линейного оператора $f$, заданного своей матрицей в некотором базисе.}%
{linear_ex10}%

\Task{Даны столбцы $a$ и $b$.  Найти столбец $c$ принадлежащий $R^3$ ортогональный $a$ такой, чтобы оболочки $<a, c>$ и $<a, b>$ совпадали.}
{linear_ex11}%

\Task{Даны столбцы $a$ и $b$.  Найти столбец $c$ принадлежащий $R^5$ ортогональный $a$ такой, чтобы оболочки $<a, c>$ и $<a, b>$ совпадали.}
{linear_ex12}%

\Task{Подпространство  $V \subset R^3$ натянуто на вектора. Ортогонализировать базис в $V$ и дополнить его до ортогонального базиса в $R^3$.}
{linear_ex13}%

\Task{Подпространство  $V \subset R^5$ натянуто на вектора. Ортогонализировать базис в $V$ и дополнить его до ортогонального базиса в $R^5$.}
{linear_ex14}%

\Task{Пусть $e$ стандартный базис в $R^3$ и даны $f_1 = (x_1; y_1; z_1)^T$, $f_2 = (x_2; y_2; z_2)^T$, $f_3 = (x_3; y_3; z_3)^T$. Даны линейные операторы $A$ и $B$, имеющие в базисе $e$ вид матриц. Также дан вектор $x = (x_0, y_0, z_0)^T$ c координатами в базисе $e$.}
{linear_ex15}%
