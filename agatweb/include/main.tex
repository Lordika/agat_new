%	----------------------		Дополнительные пакеты 		---------------------
% Русификация
\usepackage[utf8]{inputenc}
\usepackage[T2A]{fontenc}
\usepackage[russian]{babel}

% Улучшенная работа с формулами
\usepackage[intlimits]{amsmath}
\usepackage{amsfonts}

% Для программирования
\usepackage{ifthen, calc} 

% Генерация случайных чисел
\usepackage{lcg} 

% Дополнительные символы
\usepackage{textcomp} 

% Изменение расстояния междустроками
\usepackage{setspace}

% Отступ в начале главы
\usepackage{indentfirst}

% Пакеты для работы с графикой и цветом
%\usepackage[dvips]{graphicx}
\usepackage[pdftex]{color}

%Подключение гиперссылок
\usepackage[pdftex,colorlinks,linkcolor=blue,bookmarksnumbered]{hyperref}

%	----------------------		Работа с длинами 		---------------------

% Изменение геометрии страницы
\setlength{\hoffset}{-.75in}
\setlength{\voffset}{-.75in}
\addtolength{\textwidth}{1.5in}
\addtolength{\textheight}{1.5in}

%	----------------------		Создание новых счётчиков 		---------------------

% Счётчик для циклов
\newcounter{tmpa}
\newcounter{temp}
\newcounter{tmpb}
\newcounter{tmpc}

\newcounter{tmpsmart}
\newcounter{smart}
%	----------------------		Определение новых команд 		---------------------

\newcommand\ITE[3]{\ifthenelse{#1}{#2}{#3}}

% Полноразмерные формулы
\newcommand{\D}{\displaystyle}

% Скобки вокруг отрицательных переменных
\newcommand{\alg}[1]{\ifthenelse{\value{#1}<0}{{(\arabic{#1})}}{\arabic{#1}}}

%   Умный конец абзаца
\newcommand\SmartPar[1]{%
	\ifthenelse{\equal{#1}{5}}{%
		\ifthenelse{\isodd{\value{primcount}}}{}{\bigskip\par}%
	}{\ifthenelse{\equal{#1}{7}}{%
		\bigskip\par%
	}{}}%
}


\newcommand\BigPar{\bigskip\par}

\newcommand\MedPar{\medskip\par}

\newcommand\SmallPar{\smallskip\par}

\newcommand\Example[2]{%
	\refstepcounter{primcount}\theprimcount\ #1%
	\ITE{\equal{#2}{}}{\BigPar}{%
%		\SmallPar%
		\begin{center}%
			\textbf{Ответ:}\MedPar%

			#2%
		\end{center}%
		\hrulefill%
		\BigPar%
	}%
}

\newcommand\SmartPrintValue[2][1]{%
	\ToNull{smart}%
%	\ToNull{tmpsmart}%
%	\whiledo{\value{tmpsmart}<#1}{\Inc{tmpsmart}}%
%	\whiledo{\value{smart}<#2}{\Inc{smart}}%
	\Move{smart}{#2}%
%	\addtocounter{tmpsmart}{#1}%
	\ifthenelse{\equal{#1}{0}}{\ToNull{tmpsmart}}{}%
	\IfNull{smart}{}{%
		\IfNegate{smart}{%
			\Inc{tmpsmart}\IfNOne{smart}{-}{\arabic{smart}}%
		}{%
			\IfNull{tmpsmart}{}{+}\Inc{tmpsmart}\IfOne{smart}{}{\arabic{smart}}%
		}%
	}%
}

\newcommand{\Task}[2]{%
	\paragraph{{\bfseries\thetaskcount.}}#1\bigskip\par%
	\addcontentsline{toc}{section}{Задача \textnumero\thetaskcount.}%
	\input{\jobname_#2}\refstepcounter{taskcount}%
	\hfill \hyperref[sec:toc]{[К оглавлению]}.%
	\clearpage%
}

\newcommand{\ChooseTask}[1]{%
	\ifcase #1 %
		\or \Task{Найти линейную комбинацию трех матриц размерностью $3 \times 2$ ($\alpha \cdot A + \beta \cdot B + \gamma \cdot C$).}%
{add3mat_3x2}%


\Task{Перемножить две матрицы $3 \times 3$ и определить являются ли они перестановочными.}%
{mulmat_3x3x3x2}%


\Task{Возвести матрицу $2 \times 2$ в четвертую степень.}%
{degreemat_2x2x4}%


\Task{Перемножить две матрицы $4 \times 4$.}%
{mulmat_4x4x4x1}%


\Task{Перемножить матрицы: $2 \times 4$, $4 \times 2$, $2 \times 3$.}%
{mul3mat_2x4x2x3}%
%
		\or \Task{Вычислить определитель второго порядка.}%		
{det_2x2}%

\Task{Вычислить определитель третьего порядка методом Саррюса.}%
{det_3x3}%

\Task{Вычислить определитель третьего порядка разложением по строке.}%
{det_3x3_}%

\Task{Вычислить определитель четвёртого порядка.}%
{det_4x4}%

\Task{Вычислить определитель пятого порядка.}%
{det_5x5}%
%
		\or \Task{Найти обратную матрицу второго порядка через присоединённую матрицу.}%
{invert_2x2_}%

\Task{Найти обратную матрицу третьего порядка Найти обратную матрицу для матрицы 3х3 через присоединенную матрицу и методом элементарных преобразований.}%
{invert_3x3}%

\Task{Для предыдущего номера показать, что при перемножении матрицы на ее обратную получается единичная матрица.}%
{invert_3x3_}%

\Task{Найти обратную матрицу четвёртого порядка методом элементарных преобразований.}%
{invert_4x4}%

\Task{Решить матричное уравнение: $A \times X \times B = C$.}%
{invert_system_2x2}
%
		\or \Task{Решить систему линейных уравнений матричным методом.}%
{system_3x3_I}%

\Task{Решить систему линейных уравнений методом Крамера.}%
{system_3x3_K}%

\Task{Решить систему линейных уравнений методом Гаусса.}%
{system_3x3}%

\Task{Решить систему линейных уравнений методом Гаусса.}%
{system_4x4}%

\Task{Решить систему линейных уравнений методом Гаусса.}%
{system_5x5}
%
		\or \Task{Найти ранг матрицы.}%
{rang_4x4}%

\Task{Решить систему линейных уравнений методом Гаусса. Записать какое-либо частное решение.}%
{nhg_3x5_2x3_0}%

\Task{Решить систему линейных уравнений методом Гаусса. Записать какое-либо частное решение.}%
{nhg_3x4_2x3_0}%

\Task{Решить систему линейных уравнений методом Гаусса. Записать какое-либо частное решение.}%
{nhg_3x4_2x3_1}%

\Task{Решить систему линейных уравнений методом Гаусса. Записать какое-либо частное решение.}%
{nhg_3x5_2x3_1}%
%
		\or \Task{Найти $\vec {AB}$; Его направляющие косинусы; Координаты точки, которая делит $AB$ в заданном соотношении; Разложить $\vec {AB}$ по базису. Сделать рисунок проекций этого вектора на координатные плоскости.}%
{Vec_RB_3}%

\Task{Для векторов $\vec a$ и $\vec b$ найти скалярное и векторное произведения, $\cos (\widehat{\vec a, \vec b})$ и $\text{пр}_{\vec b} \vec a$.}%
{Vec_SVM}%

\Task{Для векторов $\vec a$, $\vec b$ и $\vec c$ найти двойное векторное произведение $[[\vec a, \vec b], \vec c]$.}%
{Vec_DVM}%

\Task{Определить компланарность векторов $\vec a$, $\vec b$ и $\vec c$.}%
{Vec_MMK}%

\Task{Найти смешанное произведение трёх векторов и найти объём параллелепипеда построенного на этих векторах: $\vec{OA}$, $\vec{OB}$, $\vec{OC}$.}%
{Vec_MMV}%
%
		\or \Task{Вычислить расстояние от точки до прямой.}%
{lop_delta}%
	
\Task{Вывести уравнение траектории точки, которая в каждый момент времени равноудалена от двух заданных точек.}%
{lop_eqs}%

\Task{Треугольник задан своими вершинами. Найти уравнения сторон, высот, медиан, биссектрис. Найти длины сторон и площадь треугольника.}%
{lop_triangle}%

\Task{Исследовать взаимное расположение двух прямых: если пересекаются – найти точку пересечения и угол между ними, если параллельны – найти расстояние между ними.}%
{lop_angle}%

\Task{Для данной прямой и точки, найти проекцию точки на прямую, и симметричную ей, относительно прямой точку.}%
{lop_proj}%

%
		\or \Task{Дана прямая и точка, не лежащая на этой прямой, провести через них плоскость. Ответ дать в виде общего уравнения.}%
{lp2_ex01}

\Task{Написать уравнение плоскости с заданным нормальным вектором. Найти отрезки, отсекаемые этой плоскостью на координатных осях.}%
{lap_srez}

\Task{Найти проекцию точки на плоскость. Плоскость задана уравнением в отрезках.}%
{lap_proj_dp}%

\Task{Найти точку симметричную данной точке относительно плоскости.}%
{lap_sim_dp}

\Task{Исследовать взаимное расположение двух плоскостей, где одна задана в общем виде, а другая --- через три точки. Если плоскости пересекаются --- найти  угол между ними, если параллельны --- найти расстояние между ними.}%
{lap_angle_delta}

%	\thetaskcount. Даны 3 плоскости, провести плоскость, параллельную плоскости $\alpha$ так, чтобы получился тетраэдр заданного объёма.

%	\input{\jobname_lp2_ex04} \refstepcounter{taskcount}

%	\thetaskcount. Найти угол между двумя плоскостями.

%	\input{\jobname_lp2_ex08} \refstepcounter{taskcount}
%
		\or \Task{Дан треугольник. Найти длины медиан.}%
{lp2_ex02}

\Task{Дана плоскость, треугольник и точка. Проекция треугольника на плоскость образована лучами, выходящими из точки. Найти площадь проекции.}%
{lp2_ex06}

\Task{Найти отражение луча относительно плоскости.}%
{lp2_ex07}

\Task{Найти прямую, симметричную данной, относительно плоскости.}%
{lp2_ex09}

\Task{Найти угол между двумя прямыми, заданными пересечением плоскостей.}%
{lap_angle_ll}%
%
		\or \Task{Найти собственные числа и собственные векторы матрицы 2х2. Сделать проверку.}%
{eigen_2x2_e_1}%

\Task{Найти собственные числа и собственные векторы матрицы 3х3. Сделать проверку.}%
{eigen_3x3_e_2}%

\Task{Проверить линейную зависимость собственных  векторов из предыдущих заданий.}%
{empty}

\Task{Найти собственные числа и собственные векторы транспонированной матрицы.}%
{eigen_2x2_e_1}

\Task{Найти собственные числа и собственные векторы обратной матрицы.}%
{eigen_3x3_e_2}
%
		\or \Task{Найти $\overline z$ и  $|z|$. Представить $z$ в тригонометрической и показательной форме.}%
{complex_ex01}%

\Task{Найти $z_1 + z_2$, $z_1 * z_2$, $z_1 / z_2$.}%
{complex_ex02}%

\Task{Найти $z^n$.}%
{complex_ex03}%

\Task{Найти $\sqrt[n]{z}$.}%
{complex_ex04}%

\Task{Найти корни уравнения.}%
{complex_ex05}%

%\Task{Найти решение системы уравнений.}%
%{complex_ex06}%

\Task{Выразить выражение через вещественные функции.}%
{complex_ex07}%

\Task{Выразить выражение через вещественные функции.}%
{complex_ex08}%

\Task{Выразить выражение через вещественные функции.}%
{complex_ex09}%

\Task{Выразить выражение через вещественные функции.}%
{complex_ex10}%
%
		\or \Task{Найти базис линейной оболочки строк матриц.}%
{rang_3x3}%

\Task{Найти базис линейной оболочки столбцов матрицы.}%
{rang_4x4}%

\Task{Найти базис пространства решений $Ax = 0$.}%
{rang_5x5}%

\Task{Найти базис и размерность суммы и пересечения линейных подпространств $V_1$ и $V_2$ в $R^4$, натянутых на системы векторов.}%
{basis_ex04}%

\Task{Найти базис и размерность суммы и пересечения линейных  подпространств $V_1$ и $V_2$ в $R^3$, натянутых на системы векторов.}%
{basis_ex05}%

\Task{Доказать, что матрицы образуют базис в пространстве $Mat^2(R)$ вещественных квадратных матриц порядка 2, и найти координаты матрицы $А$.}%
{basis_ex06}%

\Task{Найти матрицу перехода $C(e \to e')$, если векторы обоих базисов заданы своими координатами в некотором базисе.}%
{basis_ex07}%

\Task{Составить систему линейных уравнений, задающую линейную оболочку системы векторов.}%
{basis_ex08}%

\Task{Пусть заданы два подпространства в R4.  Показать, что $R^4 = U \oplus V$ и найти проекцию вектора $w$ на подпространство $U$ параллельно подпространству $V$.}%
{basis_ex09}%
%
		\or \Task{Линейный оператор имеет матрицу $A$. Найти его матрицу в базисе $f_1$, $f_2$.}%
{linear_ex01}%

\Task{Линейный оператор имеет матрицу $A$. Найти его матрицу в базисе $f_1$, $f_2$, $f_3$.}%
{linear_ex02}%

\Task{Линейный оператор $f$ в базисе $a_1$, $a_2$ имеет матрицу $A$.  Найти его матрицу в базисе $b_1$, $b_2$.}%
{linear_ex03}%

\Task{Линейный оператор $f$ в базисе $a_1$, $a_2$, $a_3$ имеет матрицу $A$.  Найти его матрицу в базисе $b_1$, $b_2$, $b_3$.}%
{linear_ex04}%

\Task{Доказать, что существует единственное линейное преобразование двумерного пространства, переводящее векторы $a_1$ и $a_2$ в векторы $b_1$ и $b_2$ соответственно. Найти матрицу этого преобразования в том же базисе, в котором заданы координаты векторов.}%
{linear_ex05}%

\Task{Доказать, что существует единственное линейное преобразование трёхмерного пространства, переводящее векторы $a_1$, $a_2$ и $a_3$ в векторы $b_1$, $b_2$ и $b_3$ соответственно. Найти матрицу этого преобразования в том же базисе, в котором заданы координаты векторов.}%
{linear_ex06}%

\Task{Найти базис и размерность ядра и образ линейного оператора $f$, заданного своей матрицей в некотором базисе.}%
{linear_ex07}%

\Task{Найти базис и размерность ядра и образ линейного оператора $f$, заданного своей матрицей в некотором базисе.}%
{linear_ex08}%

\Task{Найти базис и размерность ядра и образ линейного оператора $f$, заданного своей матрицей в некотором базисе.}%
{linear_ex09}%

\Task{Найти базис и размерность ядра и образ линейного оператора $f$, заданного своей матрицей в некотором базисе.}%
{linear_ex10}%

\Task{Даны столбцы $a$ и $b$.  Найти столбец $c$ принадлежащий $R^3$ ортогональный $a$ такой, чтобы оболочки $<a, c>$ и $<a, b>$ совпадали.}
{linear_ex11}%

\Task{Даны столбцы $a$ и $b$.  Найти столбец $c$ принадлежащий $R^5$ ортогональный $a$ такой, чтобы оболочки $<a, c>$ и $<a, b>$ совпадали.}
{linear_ex12}%

\Task{Подпространство  $V \subset R^3$ натянуто на вектора. Ортогонализировать базис в $V$ и дополнить его до ортогонального базиса в $R^3$.}
{linear_ex13}%

\Task{Подпространство  $V \subset R^5$ натянуто на вектора. Ортогонализировать базис в $V$ и дополнить его до ортогонального базиса в $R^5$.}
{linear_ex14}%

\Task{Пусть $e$ стандартный базис в $R^3$ и даны $f_1 = (x_1; y_1; z_1)^T$, $f_2 = (x_2; y_2; z_2)^T$, $f_3 = (x_3; y_3; z_3)^T$. Даны линейные операторы $A$ и $B$, имеющие в базисе $e$ вид матриц. Также дан вектор $x = (x_0, y_0, z_0)^T$ c координатами в базисе $e$.}
{linear_ex15}%
%
		\or 	\Task{Найти  жорданову форму и жорданов базис матрицы $A$}%
	{eigen_3x3_e_1}

	\Task{Найти собственные числа и корневые подпространства линейного оператора, заданного матрицей $A$.}%
	{eigen_3x3_e_2}

	\Task{Найти собственные числа и корневые подпространства линейного оператора, заданного матрицей $A$.}%
	{eigen_3x3_e_3}

	\Task{Найти жорданову форму, жорданов базис и минимальный многочлен для линейного оператора, заданного матрицей $A$.}%
	{jordan_ex04}

	\Task{Найти жорданову форму, жорданов базис и минимальный многочлен для линейного оператора, заданного матрицей $A$.}%
	{jordan_ex05}

	\Task{Вычислить $A^n$.}%
	{eigen_2x2_e_6}

	\Task{Вычислить $\cos A$.}%
	{eigen_2x2_e_7}

	\Task{Вычислить $e^A$.}%
	{eigen_2x2_e_8}

%
		\or \Task{Найти все нетривиальные инвариантные подпространства для линейного оператор, заданного в некотором базисе матрицей $A_f$.}%
{eigen_3x3_e_8}%

\Task{Найти базисы левого и правого ядер билинейной функции $\varphi$.}%
{eigen_3x3_k_9}%

%\Task{Найти все значения параметра $\alpha$, при которых данная квадратичная функция $Q$ положительна определена, а при которых отрицательно определена.}%
%{eigen_3x3_3}%

%\Task{Найти Найти нормальный вид в области вещественных чисел квадратичной формы $Q$.}%
%{eigen_3x3_i_10}%

\Task{Найти нормальный вид и невырожденное линейное преобразование, приводящее к этому виду, для квадратичной формы $Q$.}%
{quad_simple}%

%\Task{Для квадратичных функций $f$ и $g$ выяснить, существует ли линейное преобразование, из $f$ в $g$.}%
%{_3x3_12}%

%\Task{Существует ли линейное преобразование, переводящее квадратичную функцию $Q_1$ в квадратичную функцию $Q_2$.}%
%{_3x3_13}%
%
		\or \Task{Найти все значения параметра $\alpha$, при которых данная квадратичная функция $Q$ положительна определена.}%
{NormalForm_3x3_1_a}%

\Task{Найти все значения параметра $\alpha$, при которых данная квадратичная функция $Q$ отрицательно определена.}%
{NormalForm_3x3_1_b}%

\Task{Найти Найти нормальный вид в области вещественных чисел квадратичной формы $Q$.}%
{NormalForm_3x3_0_c}%

\Task{Для квадратичных функций $f$ и $g$ выяснить, существует ли линейное преобразование, из $f$ в $g$.}%
{NormalForm_3x3_2_d}%

\Task{Существует ли линейное преобразование, переводящее квадратичную функцию $Q_1$ в квадратичную функцию $Q_2$.}%
{NormalForm_3x3_3_e}%
%
		\or \Task{Привести уравнение к канонической форме, определить его тип, найти фокусы и эксцентриситет.}%
{crv}%
%
	\fi%
}
	
\newcommand\ToDoAns{\smallskip\par\centerline{\Large \textcolor{red}{Решения пока нет!}}}

\newcommand\Name{Nobody}

\newcommand\GetName{\textit{\inputencoding{cp866}\Name\inputencoding{utf8}}}

\newcommand\SetName[1]{\renewcommand\Name{#1}}

%\newcommand\Theme{No Theme}

%\newcommand\SetTheme[1]{\renewcommand\Theme{\textbf{#1}}}
%	----------------------		Изменение действующих команд 		---------------------

% Добавка точки к выводу переменной цикла 
\renewcommand{\thetmpa}{\textbf{\arabic{tmpa}.}}

\newcounter{taskcount}
\newcounter{primcount}
\newcounter{fifteencount}
\numberwithin{taskcount}{fifteencount}
\numberwithin{primcount}{taskcount}
\renewcommand{\theprimcount}{\textbf{Пример \arabic{primcount}.}}


% Нахождение модуля вектора и направляющих косинусов
\newcommand\VectorModule{%
	\ensuremath{\vec{a} = \{\arabic{detaa},\, \arabic{detab}\},\quad \vec{b} = \{\arabic{detba},\, \arabic{detbb}\}, \quad \vec{c} = \arabic{detac}\vec{a}\IfNegate{detbc}{}{+} \arabic{detbc}\vec{b}}%
}

% Много векторов
\newcommand\MultiVectorModule[2][\time]{%
	\reinitrand[seed=#1, first=-9, last=9]%
	\ToNull{tmpa}%
	\whiledo{\value{tmpa}<#2}{%
		\Inc{tmpa}\Example{\InitDet\VectorModule.}{}%
%		\ifthenelse{\value{tmpa}<#2}{\Inc{tmpa}\Inc{primcount} \InitDet[4] \theprimcount \InvertSystem}{} \bigskip \par%
	 }%
}

% Поиск модуля
\newcommand\SolveVectorModule[1]{%
	\ifthenelse{\equal{#1}{O}}{%
		\setcounter{detca}{\value{detaa}*\value{detac} + \value{detba}*\value{detbc}}%
		\setcounter{detcb}{\value{detab}*\value{detac} + \value{detbb}*\value{detbc}}%
		\setcounter{detcc}{\value{detca}*\value{detca} + \value{detcb}*\value{detcb}}%
		\VectorModule: $$|\vec{c}| = \sqrt{\arabic{detcc}}, \quad \cos\alpha = \frac{\arabic{detca}}{\sqrt{\arabic{detcc}}}, \quad \cos\beta = \frac{\arabic{detcb}}{\sqrt{\arabic{detcc}}}$$}{%
	}%
}

% Много поисков модулей
\newcommand\MultiSolveVectorModule[3][\time]{%
	\reinitrand[seed=#1, first=-9, last=9]%
	\ToNull{tmpa}%
	\whiledo{\value{tmpa}<#2}{%
		\Inc{tmpa}\Example{\InitDet\SolveVectorModule{#3}}{}%
	 }%
}

% Показ градусов при генерации
\newcommand\Degree[1]{%
	\addtocounter{#1}{8}%
	\ifcase \arabic{#1}%
		-180\text{\textdegree}%
		\or -150\text{\textdegree}%
		\or -135\text{\textdegree}%
		\or -120\text{\textdegree}% 
		\or -90\text{\textdegree}%
		\or -60\text{\textdegree}%
		\or -45\text{\textdegree}% 
		\or -30\text{\textdegree}%
		\or 0\text{\textdegree}%
		\or 30\text{\textdegree}% 
		\or 45\text{\textdegree}%
		\or 60\text{\textdegree}%
		\or 90\text{\textdegree}% 
		\or 120\text{\textdegree}%
		\or 135\text{\textdegree}%
		\or 150\text{\textdegree}% 
		\or 180\text{\textdegree}%
	\fi%
	\addtocounter{#1}{-8}%
}

% Показ косинуса при генерации
\newcommand\Cos[3]{%
	\addtocounter{#1}{8}%
	\ifcase \arabic{#1}%
			\setcounter{#2}{-1}\setcounter{#3}{1}%
		\or \setcounter{#2}{-2}\setcounter{#3}{3}%
		\or \setcounter{#2}{-2}\setcounter{#3}{2}%
		\or \setcounter{#2}{-2}\setcounter{#3}{1}% 
		\or \setcounter{#2}{0}\setcounter{#3}{0}%
		\or \setcounter{#2}{2}\setcounter{#3}{1}%
		\or \setcounter{#2}{2}\setcounter{#3}{2}% 
		\or \setcounter{#2}{2}\setcounter{#3}{3}%
		\or \setcounter{#2}{1}\setcounter{#3}{1}%
		\or \setcounter{#2}{2}\setcounter{#3}{3}% 
		\or \setcounter{#2}{2}\setcounter{#3}{2}%
		\or \setcounter{#2}{2}\setcounter{#3}{1}%
		\or \setcounter{#2}{0}\setcounter{#3}{0}% 
		\or \setcounter{#2}{-2}\setcounter{#3}{1}%
		\or \setcounter{#2}{-2}\setcounter{#3}{2}%
		\or \setcounter{#2}{-2}\setcounter{#3}{3}% 
		\or \setcounter{#2}{-1}\setcounter{#3}{1}%
	\fi%
	\addtocounter{#1}{-8}%
}

% Числитель с корнем
\newcommand\Chislitel[1]{%
	\ifcase \arabic{#1}%
		\or %
		\or \sqrt{2}%
		\or \sqrt{3}%
	\fi%
}

% Нахождение скалярного произведения через модуль и угол 
\newcommand\ScalarMA{%
	\ensuremath{|\vec{a}| = \arabic{detaa},\quad |\vec{b}| = \arabic{detba}, \quad \alpha = \arabic{detab}, \quad \beta = \arabic{detbb}, \quad \varphi = \Degree{detac}}%
}

% Много задач на скалярное произведение через модуль и угол
\newcommand\MultiScalarMA[2][\time]{%
	\reinitrand[seed=#1, first=-8, last=8]%
	\ToNull{tmpa}%
	\whiledo{\value{tmpa}<#2}{%
		\Inc{tmpa}\Example{\InitDet\ScalarMA.}{}%
%		\ifthenelse{\value{tmpa}<#2}{\Inc{tmpa}\Inc{primcount} \InitDet[4] \theprimcount \InvertSystem}{} \bigskip \par%
	 }%
}

% Решение задачи на скалярное произведение через модуль и угол
\newcommand\SolveScalarMA[1]{%
	\Cos{detac}{detbc}{detcc}
	\ifthenelse{\equal{#1}{O}}{%
		\setcounter{detca}{\value{detaa}*\value{detab}*\value{detba}*\value{detbb}}%
		\ScalarMA: \ensuremath{%
			(\alpha \vec{a},\, \beta \vec{b}) = \IfNull{detca}{0}{\setcounter{dettmpb}{\value{detbc}}\addtocounter{dettmpb}{2}% 
			\ifcase \arabic{dettmpb}%
					\IfNegate{detca}{\setcounter{detca}{-\value{detca}}}{-}%
					\ifthenelse{\isodd{\value{detca}}}%
						{\arabic{detca}\frac{\D\Chislitel{detcc}}{\D 2}}%
						{\setcounter{detca}{\value{detca}/2}\arabic{detca}\Chislitel{detcc}}%
				\or \setcounter{detca}{-\value{detca}}\arabic{detca}%
				\or 0%
				\or \arabic{detca}%
				\or \IfNegate{detca}{-\setcounter{detca}{-\value{detca}}}{}\ifthenelse{\isodd{\value{detca}}}{\arabic{detca}\frac{\D\Chislitel{detcc}}{\D 2}}{\setcounter{detca}{\value{detca}/2}\arabic{detca}\Chislitel{detcc}}%
			\fi}%
		}%
	}{}%
}

% Много решений задач на скалярное произведение через модуль и угол
\newcommand\MultiSolveScalarMA[3][\time]{%
	\reinitrand[seed=#1, first=-8, last=8]%
	\ToNull{tmpa}%
	\whiledo{\value{tmpa}<#2}{%
		\Inc{tmpa}\Example{\InitDet\SolveScalarMA{#3}}{}%
	 }%
}

% Нахождение скалярного произведения через координаты
\newcommand\ScalarDC{%
	\ensuremath{\vec{a} = \{\arabic{detaa},\, \arabic{detab},\, \arabic{detac}\},\quad \vec{b} = \{\arabic{detba},\, \arabic{detbb},\, \arabic{detbc}\}}%
}

% Много задач на скалярное произведение через координаты
\newcommand\MultiScalarDC[2][\time]{%
	\reinitrand[seed=#1, first=-9, last=9]%
	\ToNull{tmpa}%
	\whiledo{\value{tmpa}<#2}{%
		\Inc{tmpa}\Example{\InitDet\ScalarDC.}{}%
%		\ifthenelse{\value{tmpa}<#2}{\Inc{tmpa}\Inc{primcount} \InitDet[4] \theprimcount \InvertSystem}{} \bigskip \par%
	 }%
}

% Решение задачи на скалярное произведение через координаты
\newcommand\SolveScalarDC[1]{%
	\ifthenelse{\equal{#1}{O}}{%
		\setcounter{detca}{\value{detaa}*\value{detaa} + \value{detab}*\value{detab} + \value{detac}*\value{detac}}%
		\setcounter{detcb}{\value{detba}*\value{detba} + \value{detbb}*\value{detbb} + \value{detbc}*\value{detbc}}%
		\Multi{detca}{detca}{detcb}
		\setcounter{detcc}{\value{detaa}*\value{detba} + \value{detab}*\value{detbb} + \value{detac}*\value{detbc}}%
		\ScalarDC: \ensuremath{\cos\varphi = \frac{\D\arabic{detcc}}{\D\sqrt{\arabic{detca}}}}}{%
	}%
}

% Много решений задач на скалярное произведение через координаты
\newcommand\MultiSolveScalarDC[3][\time]{%
	\reinitrand[seed=#1, first=-9, last=9]%
	\ToNull{tmpa}%
	\whiledo{\value{tmpa}<#2}{%
		\Inc{tmpa}\Example{\InitDet\SolveScalarDC{#3}}{}%
	 }%
}

% Нахождение проекции одного вектора на другой 
\newcommand\VectorProjection{%
	\ensuremath{\vec{a} = \{\arabic{detaa},\, \arabic{detab},\, \arabic{detac}\},\quad \vec{b} = \{\arabic{detba},\, \arabic{detbb},\, \arabic{detbc}\}}%
}

% Много задач на проекции векторов
\newcommand\MultiVectorProjection[2][\time]{%
	\reinitrand[seed=#1, first=-9, last=9]%
	\ToNull{tmpa}%
	\whiledo{\value{tmpa}<#2}{%
		\Inc{tmpa}\Example{\InitDet\VectorProjection.}{}%
%		\ifthenelse{\value{tmpa}<#2}{\Inc{tmpa}\Inc{primcount} \InitDet[4] \theprimcount \InvertSystem}{} \bigskip \par%
	 }%
}

% Решение задачи на проекции векторов
\newcommand\SolveVectorProjection[1]{%
	\ifthenelse{\equal{#1}{O}}{%
		\setcounter{detca}{\value{detaa}*\value{detaa} + \value{detab}*\value{detab} + \value{detac}*\value{detac}}%
		\setcounter{detcc}{\value{detaa}*\value{detba} + \value{detab}*\value{detbb} + \value{detac}*\value{detbc}}%
		\VectorProjection: \ensuremath{\text{пр}_{\vec{a}} \vec{b} = \frac{\D\arabic{detcc}}{\D\sqrt{\arabic{detca}}}}}{%
	}%
}

% Много решений задач на проекции векторов
\newcommand\MultiSolveVectorProjection[3][\time]{%
	\reinitrand[seed=#1, first=-9, last=9]%
	\ToNull{tmpa}%
	\whiledo{\value{tmpa}<#2}{%
		\Inc{tmpa}\Example{\InitDet\SolveVectorProjection{#3}}{}%
	 }%
}

% Нахождение проекций треугольника
\newcommand\TriangleProjection{%
	\ensuremath{A = (\arabic{detaa},\, \arabic{detab}),\quad B = (\arabic{detba},\, \arabic{detbb}),\quad C = (\arabic{detca},\, \arabic{detcb})}%
}

% Много задач на проекции треугольника
\newcommand\MultiTriangleProjection[2][\time]{%
	\reinitrand[seed=#1, first=-9, last=9]%
	\ToNull{tmpa}%
	\whiledo{\value{tmpa}<#2}{%
		\Inc{tmpa}\Example{\InitDet[4]\TriangleProjection.}{}%
%		\ifthenelse{\value{tmpa}<#2}{\Inc{tmpa}\Inc{primcount} \InitDet[4] \theprimcount \InvertSystem}{} \bigskip \par%
	 }%
}

% Решение задачи на проекции треугольника
\newcommand\SolveTriangleProjection[1]{%
	\ifthenelse{\equal{#1}{O}}{%
		\setcounter{detac}{\value{detca}-\value{detaa}}%
		\setcounter{detad}{\value{detcb}-\value{detab}}%
		\setcounter{detbc}{\value{detba}-\value{detca}}%
		\setcounter{detbd}{\value{detbb}-\value{detcb}}%
		\setcounter{detcc}{\value{detaa}-\value{detba}}%
		\setcounter{detcd}{\value{detab}-\value{detbb}}%
		\setcounter{detda}{\value{detac}*\value{detac} + \value{detad}*\value{detad}}%
		\setcounter{detdb}{\value{detbc}*\value{detbc} + \value{detbd}*\value{detbd}}%
		\setcounter{detdc}{\value{detcc}*\value{detcc} + \value{detcd}*\value{detcd}}%
		\TriangleProjection:%
		\setcounter{detdd}{\value{detac}*\value{detbc} + \value{detad}*\value{detbd}}%
		$$\text{пр}_{CB} AC  = \frac{\arabic{detdd}}{\sqrt{\arabic{detdb}}}%
		\setcounter{detdd}{\value{detcc}*\value{detbc} + \value{detcd}*\value{detbd}}%
		\quad \text{пр}_{BA} CB  = \frac{\arabic{detdd}}{\sqrt{\arabic{detdc}}}%
		\setcounter{detdd}{\value{detcc}*\value{detac} + \value{detcd}*\value{detad}}%
		\quad \text{пр}_{AC} BA  = \frac{\arabic{detdd}}{\sqrt{\arabic{detda}}}$$}{%
	}%
}

% Много решений задач на проекции треугольника
\newcommand\MultiSolveTriangleProjection[3][\time]{%
	\reinitrand[seed=#1, first=-9, last=9]%
	\ToNull{tmpa}%
	\whiledo{\value{tmpa}<#2}{%
		\Inc{tmpa}\Example{\InitDet[4]\SolveTriangleProjection{#3}}{}%
	 }%
}

\newcommand\VectorMultiply{%
	\ensuremath{\vec a = {\{\arabic{detaa},\, \arabic{detab},\, \arabic{detac}\}},\quad%
				\vec b = {\{\arabic{detba},\, \arabic{detbb},\, \arabic{detbc}\}}%
	}%
}

% Много задач на векторное произведение
\newcommand\MultiVectorMultiply[2][\time]{%
	\reinitrand[seed=#1, first=-9, last=9]%
	\ToNull{tmpa}%
	\whiledo{\value{tmpa}<#2}{%
		\Inc{tmpa}\Example{\InitDet\VectorMultiply.}{}%
%		\ifthenelse{\value{tmpa}<#2}{\Inc{tmpa}\Inc{primcount} \InitDet[4] \theprimcount \InvertSystem}{} \bigskip \par%
	 }%
}

% Решение задачи на векторное произведение
\newcommand\SolveVectorMultiply[1]{%
	\ifthenelse{\equal{#1}{O}}{%
		\setcounter{detca}{\value{detab}*\value{detbc}-\value{detac}*\value{detbb}}%
		\setcounter{detcb}{\value{detac}*\value{detba}-\value{detaa}*\value{detbc}}%
		\setcounter{detcc}{\value{detaa}*\value{detbb}-\value{detba}*\value{detab}}%
%		\setcounter{dettmpb}{\value{detca}*\value{detca} + \value{detcb}*\value{detcb} + \value{detcc}*\value{detcc}}%
		\VectorMultiply:%
		$$[\vec a, \vec b] = \{\arabic{detca},\, \arabic{detcb},\, \arabic{detcc}\}.$$%
	}{}%
}

% Много решений задач на векторное произведение
\newcommand\MultiSolveVectorMultiply[3][\time]{%
	\reinitrand[seed=#1, first=-9, last=9]%
	\ToNull{tmpa}%
	\whiledo{\value{tmpa}<#2}{%
		\Inc{tmpa}\Example{\InitDet\SolveVectorMultiply{#3}}{}%
	 }%
}

\newcommand\ScalarVectorMultiply{%
	\VectorMultiply
}

% Много задач на векторное произведение
\newcommand\MultiScalarVectorMultiply[2][\time]{%
	\reinitrand[seed=#1, first=-9, last=9]%
	\ToNull{tmpa}%
	\whiledo{\value{tmpa}<#2}{%
		\Inc{tmpa}\Example{\InitDet\ScalarVectorMultiply.}{}%
%		\ifthenelse{\value{tmpa}<#2}{\Inc{tmpa}\Inc{primcount} \InitDet[4] \theprimcount \InvertSystem}{} \bigskip \par%
	 }%
}

% Решение задачи на векторное произведение
\newcommand\SolveScalarVectorMultiply[1]{%
	\ifthenelse{\equal{#1}{O}}{%
		\ScalarVectorMultiply:%
		\setcounter{detca}{\value{detaa}*\value{detba}+\value{detab}*\value{detbb}+\value{detac}*\value{detbc}}%
		\setcounter{detcb}{\value{detaa}*\value{detaa}+\value{detab}*\value{detab}+\value{detac}*\value{detac}}%
		\setcounter{detcc}{\value{detba}*\value{detba}+\value{detbb}*\value{detbb}+\value{detbc}*\value{detbc}}%
		$$(\vec a, \vec b) = \arabic{detca} \qquad \cos\widehat{\vec a, \vec b} = \frac{\arabic{detca}}{\sqrt{\arabic{detcb}}\cdot\sqrt{\arabic{detcc}}},$$
		\setcounter{detca}{\value{detab}*\value{detbc}-\value{detac}*\value{detbb}}%
		\setcounter{detcb}{\value{detac}*\value{detba}-\value{detaa}*\value{detbc}}%
		\setcounter{detcc}{\value{detaa}*\value{detbb}-\value{detba}*\value{detab}}%
%		\setcounter{dettmpb}{\value{detca}*\value{detca} + \value{detcb}*\value{detcb} + \value{detcc}*\value{detcc}}%
		$$[\vec a, \vec b] = \{\arabic{detca},\, \arabic{detcb},\, \arabic{detcc}\}.$$%
	}{}%
}

% Много решений задач на векторное произведение
\newcommand\MultiSolveScalarVectorMultiply[3][\time]{%
	\reinitrand[seed=#1, first=-9, last=9]%
	\ToNull{tmpa}%
	\whiledo{\value{tmpa}<#2}{%
		\Inc{tmpa}\Example{\InitDet\SolveScalarVectorMultiply{#3}}{}%
	 }%
}

% Двойное векторное произведение
\newcommand\DoubleVectorMultiply{%
	\ensuremath{\vec a = {\{\arabic{detaa},\, \arabic{detab},\, \arabic{detac}\}},\quad%
				\vec b = {\{\arabic{detba},\, \arabic{detbb},\, \arabic{detbc}\}},\quad%
				\vec c = {\{\arabic{detca},\, \arabic{detcb},\, \arabic{detcc}\}}%
	}%
}

% Много задач на двойное векторное произведение
\newcommand\MultiDoubleVectorMultiply[2][\time]{%
	\reinitrand[seed=#1, first=-9, last=9]%
	\ToNull{tmpa}%
	\whiledo{\value{tmpa}<#2}{%
		\Inc{tmpa}\Example{\InitDet[4]\DoubleVectorMultiply.}{}%
%		\ifthenelse{\value{tmpa}<#2}{\Inc{tmpa}\Inc{primcount} \InitDet[4] \theprimcount \InvertSystem}{} \bigskip \par%
	 }%
}           

% Решение задачи на двойное векторное произведение
\newcommand\SolveDoubleVectorMultiply[1]{%
	\ifthenelse{\equal{#1}{O}}{%
		\setcounter{detda}{\value{detab}*\value{detbc}-\value{detac}*\value{detbb}}%
		\setcounter{detdb}{\value{detac}*\value{detba}-\value{detaa}*\value{detbc}}%
		\setcounter{detdc}{\value{detaa}*\value{detbb}-\value{detba}*\value{detab}}%
		\setcounter{detad}{\value{detdb}*\value{detcc}-\value{detdc}*\value{detcb}}%
		\setcounter{detbd}{\value{detdc}*\value{detca}-\value{detda}*\value{detcc}}%
		\setcounter{detcd}{\value{detda}*\value{detcb}-\value{detca}*\value{detdb}}%
%		\setcounter{dettmpb}{\value{detca}*\value{detca} + \value{detcb}*\value{detcb} + \value{detcc}*\value{detcc}}%
		\DoubleVectorMultiply:%
		$$[[\vec a, \vec b], \vec c] = \{\arabic{detad},\, \arabic{detbd},\, \arabic{detcd}\}.$$%
	}{}%
}

% Много решений задач на двойное векторное произведение
\newcommand\MultiSolveDoubleVectorMultiply[3][\time]{%
	\reinitrand[seed=#1, first=-9, last=9]%
	\ToNull{tmpa}%
	\whiledo{\value{tmpa}<#2}{%
		\Inc{tmpa}\Inc{primcount}\InitDet[4]\theprimcount\ \SolveDoubleVectorMultiply{#3} \BigPar%
	 }%
}

% Смешанное произведение
\newcommand\MixedMultiply{%
	\ensuremath{\vec a = {\{\arabic{detaa},\, \arabic{detab},\, \arabic{detac}\}},\quad%
				\vec b = {\{\arabic{detba},\, \arabic{detbb},\, \arabic{detbc}\}},\quad%
				\vec c = {\{\arabic{detca},\, \arabic{detcb},\, \arabic{detcc}\}}%
	}%
}

% Много задач на смешанное произведение
\newcommand\MultiMixedMultiply[2][\time]{%
	\reinitrand[seed=#1, first=-9, last=9]%
	\ToNull{tmpa}%
	\whiledo{\value{tmpa}<#2}{%
		\Inc{tmpa}\Example{\InitDet[4]\MixedMultiply.}{}%
%		\ifthenelse{\value{tmpa}<#2}{\Inc{tmpa}\Inc{primcount} \InitDet[4] \theprimcount \InvertSystem}{} \bigskip \par%
	 }%
}           

% Решение задачи на смешанное произведение
\newcommand\SolveMixedMultiply[1]{%
	\ifthenelse{\equal{#1}{O}}{%
		\setcounter{detda}{\value{detaa}*\value{detbb}*\value{detcc}+\value{detab}*\value{detbc}*\value{detca}+\value{detba}*\value{detcb}*\value{detac} - \value{detac}*\value{detbb}*\value{detca} - \value{detaa}*\value{detbc}*\value{detcb} - \value{detab}*\value{detba}*\value{detcc}}%
%		\setcounter{dettmpb}{\value{detca}*\value{detca} + \value{detcb}*\value{detcb} + \value{detcc}*\value{detcc}}%
		\MixedMultiply:%
		$$([\vec a, \vec b], \vec c) = \arabic{detda}.$$%
	}{}%
}

% Много решений задач на смешанное произведение
\newcommand\MultiSolveMixedMultiply[3][\time]{%
	\reinitrand[seed=#1, first=-9, last=9]%
	\ToNull{tmpa}%
	\whiledo{\value{tmpa}<#2}{%
		\Inc{tmpa}\Example{\InitDet[4]\SolveMixedMultiply{#3}}{}%
	 }%
}

% Площадь треугольника
\newcommand\VectorMultiplyS{%
	\ensuremath{A = {(\arabic{detaa},\, \arabic{detab},\, \arabic{detac})},\quad%
				B = {(\arabic{detba},\, \arabic{detbb},\, \arabic{detbc})},\quad%
				C = {(\arabic{detca},\, \arabic{detcb},\, \arabic{detcc})}
	}%
}

% Много задач на площадь треугольника через векторное произведение
\newcommand\MultiVectorMultiplyS[2][\time]{%
	\reinitrand[seed=#1, first=-9, last=9]%
	\ToNull{tmpa}%
	\whiledo{\value{tmpa}<#2}{%
		\Inc{tmpa}\Example{\InitDet[4]\VectorMultiplyS.}{}%
%		\ifthenelse{\value{tmpa}<#2}{\Inc{tmpa}\Inc{primcount} \InitDet[4] \theprimcount \InvertSystem}{} \bigskip \par%
	 }%
}

% Решение задачи на площадь треугольника через векторное произведение
\newcommand\SolveVectorMultiplyS[1]{%
	\ifthenelse{\equal{#1}{O}}{%
		\VectorMultiplyS:%
		\setcounter{detda}{\value{detba}-\value{detaa}}%
		\setcounter{detdb}{\value{detbb}-\value{detab}}%
		\setcounter{detdc}{\value{detbc}-\value{detac}}%
		\setcounter{detad}{\value{detca}-\value{detaa}}%
		\setcounter{detbd}{\value{detcb}-\value{detab}}%
		\setcounter{detcd}{\value{detcc}-\value{detac}}%
		\setcounter{detaa}{\value{detdb}*\value{detcd}-\value{detdc}*\value{detbd}}%
		\setcounter{detab}{\value{detdc}*\value{detad}-\value{detda}*\value{detcd}}%
		\setcounter{detac}{\value{detda}*\value{detbd}-\value{detdb}*\value{detad}}%
		\setcounter{detdd}{\value{detaa}*\value{detaa} + \value{detab}*\value{detab} + \value{detac}*\value{detac}}%
		$$S_{\triangle ABC} = \frac12 |[\vec{AB}, \vec AC]| = \frac{\sqrt{\arabic{detdd}}}{2}.$$%
	}{}%
}

% Много решений задач на площадь треугольника через векторное произведение
\newcommand\MultiSolveVectorMultiplyS[3][\time]{%
	\reinitrand[seed=#1, first=-9, last=9]%
	\ToNull{tmpa}%
	\whiledo{\value{tmpa}<#2}{%
		\Inc{tmpa}\Example{\InitDet[4]\SolveVectorMultiplyS{#3}}{}%
	 }%
}

% Компланарность векторов 
\newcommand\MixedMultiplyK{%
	\ensuremath{\vec a = {\{\arabic{detaa},\, \arabic{detab},\, \arabic{detac}\}},\quad%
				\vec b = {\{\arabic{detba},\, \arabic{detbb},\, \arabic{detbc}\}},\quad%
				\vec c = {\{\arabic{detca},\, \arabic{detcb},\, \arabic{detcc}\}}%
	}%
}

% Много задач на компланарность векторов 
\newcommand\MultiMixedMultiplyK[2][\time]{%
	\reinitrand[seed=#1, first=-3, last=3]%
	\ToNull{tmpa}%
	\whiledo{\value{tmpa}<#2}{%
		\Inc{tmpa}\Example{\InitDet[4]\MixedMultiplyK.}{}%
%		\ifthenelse{\value{tmpa}<#2}{\Inc{tmpa}\Inc{primcount} \InitDet[4] \theprimcount \InvertSystem}{} \bigskip \par%
	 }%
}           

% Решение задачи на компланарность векторов 
\newcommand\SolveMixedMultiplyK[1]{%
	\ifthenelse{\equal{#1}{O}}{%
		\setcounter{detdd}{\value{detaa}*\value{detbb}*\value{detcc}+\value{detab}*\value{detbc}*\value{detca}+\value{detba}*\value{detcb}*\value{detac} - \value{detac}*\value{detbb}*\value{detca} - \value{detaa}*\value{detbc}*\value{detcb} - \value{detab}*\value{detba}*\value{detcc}}%
		\MixedMultiplyK:%
		$$([\vec a, \vec b], \vec c) = \arabic{detdd} \Longrightarrow \text{ вектора %
			\IfNull{detdd}{}{не } компланарны
		}.$$%
	}{}%
}

% Много решений задач на компланарность векторов 
\newcommand\MultiSolveMixedMultiplyK[3][\time]{%
	\reinitrand[seed=#1, first=-3, last=3]%
	\ToNull{tmpa}%
	\whiledo{\value{tmpa}<#2}{%
		\Inc{tmpa}\Example{\InitDet[4]\SolveMixedMultiplyK{#3}}{}%
	 }%
}

% Объём через смешанное произведение
\newcommand\MixedMultiplyV{%
	\ensuremath{O = {(\arabic{detda},\, \arabic{detdb},\, \arabic{detdc})},\,%
				A = {(\arabic{detaa},\, \arabic{detab},\, \arabic{detac})},\,%
				B = {(\arabic{detba},\, \arabic{detbb},\, \arabic{detbc})},\,%
				C = {(\arabic{detca},\, \arabic{detcb},\, \arabic{detcc})}%
	}%
}

% Много задач на объём через смешанное произведение
\newcommand\MultiMixedMultiplyV[2][\time]{%
	\reinitrand[seed=#1, first=-9, last=9]%
	\ToNull{tmpa}%
	\whiledo{\value{tmpa}<#2}{%
		\Inc{tmpa}\Example{\InitDet[4]\MixedMultiplyV.}{}%
%		\ifthenelse{\value{tmpa}<#2}{\Inc{tmpa}\Inc{primcount} \InitDet[4] \theprimcount \InvertSystem}{} \bigskip \par%
	}%
}           

% Решение задачи на объём через смешанное произведение
\newcommand\SolveMixedMultiplyV[1]{%
	\ifthenelse{\equal{#1}{O}}{%
		\MixedMultiplyV:%
		\setcounter{detaa}{\value{detaa}-\value{detda}}%
		\setcounter{detab}{\value{detab}-\value{detdb}}%
		\setcounter{detac}{\value{detac}-\value{detdc}}%
		\setcounter{detba}{\value{detba}-\value{detda}}%
		\setcounter{detbb}{\value{detbb}-\value{detdb}}%
		\setcounter{detbc}{\value{detbc}-\value{detdc}}%
		\setcounter{detca}{\value{detca}-\value{detda}}%
		\setcounter{detcb}{\value{detcb}-\value{detdb}}%
		\setcounter{detcc}{\value{detcc}-\value{detdc}}%
		\setcounter{detdd}{\value{detaa}*\value{detbb}*\value{detcc}+\value{detab}*\value{detbc}*\value{detca}+\value{detba}*\value{detcb}*\value{detac} - \value{detac}*\value{detbb}*\value{detca} - \value{detaa}*\value{detbc}*\value{detcb} - \value{detab}*\value{detba}*\value{detcc}}%
%		\setcounter{dettmpb}{\value{detca}*\value{detca} + \value{detcb}*\value{detcb} + \value{detcc}*\value{detcc}}%
		\IfNegate{detdd}{\setcounter{detdd}{-\value{detdd}}}{}
		$$V = |([\vec {OA}, \vec {OB}], \vec {OC})| = \arabic{detdd}.$$%
	}{}%
}

% Много решений задач на объём через смешанное произведение
\newcommand\MultiSolveMixedMultiplyV[3][\time]{%
	\reinitrand[seed=#1, first=-9, last=9]%
	\ToNull{tmpa}%
	\whiledo{\value{tmpa}<#2}{%
		\Inc{tmpa}\Example{\InitDet[4]\SolveMixedMultiplyV{#3}}{}%
	 }%
}

\newcommand\ARG{\mathrm{ARG}}
\newcommand\TheComplex[2]{\arabic{#1} \IfNegate{#2}{}{+} \arabic{#2}i}

% Нахождение модуля и аргумента комплексного числа
\newcommand\ComplexOne{%
	\ensuremath{z = \TheComplex{detaa}{detab}}%
}

\newcommand\MultiComplexOne[2][\time]{%
	\reinitrand[seed=#1, first=-9, last=9]%
	\ToNull{tmpa}%
	\whiledo{\value{tmpa}<#2}{%
		\Inc{tmpa}\Inc{primcount} \InitDet \theprimcount\ \ComplexOne. %\qquad%
		\\[.5cm] \par
%		\ifthenelse{\value{tmpa}<#2}{\Inc{tmpa}\Inc{primcount} \InitDet[4] \theprimcount \InvertSystem}{} \\[.5cm] \par%
	 }%
}

\newcommand\SolveComplexOne[1]{%
	\ifthenelse{\equal{#1}{O}}{%
		\setcounter{detca}{\value{detaa}*\value{detaa} + \value{detab}*\value{detab}}%
		\ComplexOne: %
		$$%
			|z| = r = \sqrt{\arabic{detca}}, \quad \ARG(z) = \varphi = \arctg\frac{\arabic{detab}}{\arabic{detaa}}%
		$$%
	}{%
	}%
}

\newcommand\MultiSolveComplexOne[3][\time]{%
	\reinitrand[seed=#1, first=-9, last=9]%
	\ToNull{tmpa}%
	\whiledo{\value{tmpa}<#2}{%
		\Inc{tmpa}\Inc{primcount}\theprimcount\ \InitDet \SolveComplexOne{#3} \par%\\[.5cm] \par%
	 }%
}

% Нахождение суммы, произведения и частного двух комплексных чисел
\newcommand\ComplexOps{%
	\ensuremath{z_1 = \TheComplex{detaa}{detab}, \quad z_2 = \TheComplex{detba}{detbb}}%
}

\newcommand\MultiComplexOps[2][\time]{%
	\reinitrand[seed=#1, first=-9, last=9]%
	\ToNull{tmpa}%
	\whiledo{\value{tmpa}<#2}{%
		\Inc{tmpa}\Inc{primcount} \InitDet \theprimcount\ \ComplexOps. %\qquad%
		\\[.5cm] \par
%		\ifthenelse{\value{tmpa}<#2}{\Inc{tmpa}\Inc{primcount} \InitDet[4] \theprimcount \InvertSystem}{} \\[.5cm] \par%
	 }%
}

\newcommand\SolveComplexOps[1]{%
	\ifthenelse{\equal{#1}{O}}{%
		\ComplexOps: 
		$$
			\setcounter{detac}{\value{detaa} + \value{detba}}%
			\setcounter{detbc}{\value{detab} + \value{detbb}}%
			z_1 + z_2 = \TheComplex{detac}{detbc},%
			\setcounter{detac}{\value{detaa}*\value{detba} - \value{detab}*\value{detbb}}%
			\setcounter{detbc}{\value{detaa}*\value{detbb} + \value{detba}*\value{detab}}%
			\quad z_1 \cdot z_2 = \TheComplex{detac}{detbc},%
			\setcounter{detac}{\value{detaa}*\value{detba} + \value{detab}*\value{detbb}}%
			\setcounter{detbc}{-\value{detaa}*\value{detbb} + \value{detba}*\value{detab}}%
			\setcounter{detcc}{\value{detba}*\value{detba} + \value{detbb}*\value{detbb}}%
			\quad \frac{z_1}{z_2} = \frac{\TheComplex{detac}{detbc}}{\arabic{detcc}}.%
		$$}{%
	}%
}

\newcommand\MultiSolveComplexOps[3][\time]{%
	\reinitrand[seed=#1, first=-9, last=9]%
	\ToNull{tmpa}%
	\whiledo{\value{tmpa}<#2}{%
		\Inc{tmpa}\Inc{primcount}\theprimcount\ \InitDet \SolveComplexOps{#3} \par%\\[.5cm] \par%
	 }%
}

% Решение кубического уравнения
\newcommand\ComplexCubic{%
	\ensuremath{%
		\IfNegate{detac}{\Inverse{detac}}{}
		\setcounter{detac}{\value{detac} / 2}
		\IfNull{detaa}{\Dec{detaa}}{}%
		\IfNull{detab}{\Inc{detab}}{}%
		\IfNull{detac}{\Inc{detac}}{}%
		\IfOne{detac}{\Inc{detac}}{}
		\setcounter{detbc}{\value{detac} * \value{detab} * \value{detab}}%
		\setcounter{detca}{- 2 * \value{detac} * \value{detab} * \value{detba} - \value{detaa} * \value{detab} * \value{detab}}%
		\setcounter{detcb}{\value{detac} * \value{detba} * \value{detba} + \value{detac} * \value{detbb} * \value{detbb} + 2 * \value{detaa} * \value{detab} * \value{detba}}%
		\setcounter{detcc}{- \value{detaa} * \value{detba} * \value{detba} - \value{detaa} * \value{detbb} * \value{detbb}}%
		\SmartPrintValue[0]{detbc} x^3 \SmartPrintValue{detca} \IfNull{detca}{}{x^2} \SmartPrintValue{detcb}\IfNull{detcb}{}{x} \SmartPrintValue{detcc} = 0%
	}%
}

\newcommand\MultiComplexCubic[2][\time]{%
	\reinitrand[seed=#1, first=-9, last=9]%
	\ToNull{tmpa}%
	\whiledo{\value{tmpa}<#2}{%
		\Inc{tmpa}\Inc{primcount} \InitDet \theprimcount\ \ComplexCubic. %\qquad%
		\\[.5cm] \par
%		\ifthenelse{\value{tmpa}<#2}{\Inc{tmpa}\Inc{primcount} \InitDet[4] \theprimcount \InvertSystem}{} \\[.5cm] \par%
	 }%
}

\newcommand\SolveComplexCubic[1]{%
	\ifthenelse{\equal{#1}{O}}{%
		\IfNegate{detac}{\Inverse{detac}}{}
		\setcounter{detac}{\value{detac} / 2}
		\IfNull{detaa}{\Dec{detaa}}{}%
		\IfNull{detab}{\Inc{detab}}{}%
		\IfNull{detac}{\Inc{detac}}{}%
		\IfOne{detac}{\Inc{detac}}{}
		\ComplexCubic: $$x_1 = \frac{\arabic{detaa}}{\arabic{detac}}, \quad x_{2,3} = \IfOne{detab}{\arabic{detba}\pm\alg{detbb}i}{\frac{\arabic{detba}\pm\alg{detbb}i}{\arabic{detab}}}$$}{%
	}%
}

\newcommand\MultiSolveComplexCubic[3][\time]{%
	\reinitrand[seed=#1, first=-9, last=9]%
	\ToNull{tmpa}%
	\whiledo{\value{tmpa}<#2}{%
		\Inc{tmpa}\Inc{primcount}\theprimcount\ \InitDet \SolveComplexCubic{#3} \par%\\[.5cm] \par%
	 }%
}

% Нахождение натурального логарифма комплексного числа
\newcommand\ComplexLn{%
	\ensuremath{z = \TheComplex{detaa}{detab}}%
}

\newcommand\MultiComplexLn[2][\time]{%
	\reinitrand[seed=#1, first=-9, last=9]%
	\ToNull{tmpa}%
	\whiledo{\value{tmpa}<#2}{%
		\Inc{tmpa}\Inc{primcount} \InitDet \theprimcount\ \ComplexLn. %\qquad%
		\\[.5cm] \par
%		\ifthenelse{\value{tmpa}<#2}{\Inc{tmpa}\Inc{primcount} \InitDet[4] \theprimcount \InvertSystem}{} \\[.5cm] \par%
	 }%
}

\newcommand\SolveComplexLn[1]{%
	\ifthenelse{\equal{#1}{O}}{%
		\setcounter{detca}{\value{detaa}*\value{detaa} + \value{detab}*\value{detab}}%
		\ComplexLn: %
		$$%
			\ln(z) = \frac12 \ln\arabic{detca} + i(\arctg\frac{\arabic{detab}}{\arabic{detaa}}+2\pi k), \quad k \in Z.%
		$$%
	}{%
	}%
}

\newcommand\MultiSolveComplexLn[3][\time]{%
	\reinitrand[seed=#1, first=-9, last=9]%
	\ToNull{tmpa}%
	\whiledo{\value{tmpa}<#2}{%
		\Inc{tmpa}\Inc{primcount}\theprimcount\ \InitDet \SolveComplexLn{#3} \par%\\[.5cm] \par%
	 }%
}

% Нахождение Sh(z)
\newcommand\ComplexSh{%
	\ensuremath{\sh(\TheComplex{detaa}{detab})}%
}

\newcommand\MultiComplexSh[2][\time]{%
	\reinitrand[seed=#1, first=-9, last=9]%
	\ToNull{tmpa}%
	\whiledo{\value{tmpa}<#2}{%
		\Inc{tmpa}\Inc{primcount} \InitDet \theprimcount\ \ComplexSh. %\qquad%
		\\[.5cm] \par
%		\ifthenelse{\value{tmpa}<#2}{\Inc{tmpa}\Inc{primcount} \InitDet[4] \theprimcount \InvertSystem}{} \\[.5cm] \par%
	 }%
}

\newcommand\SolveComplexSh[1]{%
	\ifthenelse{\equal{#1}{O}}{%
		$$%
			\ComplexSh = \sh(\arabic{detaa})\cos(\arabic{detab}) + i \sin(\arabic{detab})\ch(\arabic{detaa}).
		$$%
	}{%
	}%
}

\newcommand\MultiSolveComplexSh[3][\time]{%
	\reinitrand[seed=#1, first=-9, last=9]%
	\ToNull{tmpa}%
	\whiledo{\value{tmpa}<#2}{%
		\Inc{tmpa}\Inc{primcount}\theprimcount\ \InitDet \SolveComplexSh{#3} \par%\\[.5cm] \par%
	 }%
}

% Нахождение Ch(z)
\newcommand\ComplexCh{%
	\ensuremath{\ch(\TheComplex{detaa}{detab})}%
}

\newcommand\MultiComplexCh[2][\time]{%
	\reinitrand[seed=#1, first=-9, last=9]%
	\ToNull{tmpa}%
	\whiledo{\value{tmpa}<#2}{%
		\Inc{tmpa}\Inc{primcount} \InitDet \theprimcount\ \ComplexCh. %\qquad%
		\\[.5cm] \par
%		\ifthenelse{\value{tmpa}<#2}{\Inc{tmpa}\Inc{primcount} \InitDet[4] \theprimcount \InvertSystem}{} \\[.5cm] \par%
	 }%
}

\newcommand\SolveComplexCh[1]{%
	\ifthenelse{\equal{#1}{O}}{%
		$$%
			\ComplexCh = \ch(\arabic{detaa})\cos(\arabic{detab}) + i \sin(\arabic{detab})\sh(\arabic{detaa}).
		$$%
	}{%
	}%
}

\newcommand\MultiSolveComplexCh[3][\time]{%
	\reinitrand[seed=#1, first=-9, last=9]%
	\ToNull{tmpa}%
	\whiledo{\value{tmpa}<#2}{%
		\Inc{tmpa}\Inc{primcount}\theprimcount\ \InitDet \SolveComplexCh{#3} \par%\\[.5cm] \par%
	 }%
}

% Взятие корня комплексного числа
\newcommand\ComplexSqrt{%
	\IfNegate{detac}{\Inverse{detac}}{}%
	\ensuremath{\sqrt[\arabic{detac}]{\TheComplex{detaa}{detab}}}%
}

\newcommand\MultiComplexSqrt[2][\time]{%
	\reinitrand[seed=#1, first=-9, last=9]%
	\ToNull{tmpa}%
	\whiledo{\value{tmpa}<#2}{%
		\Inc{tmpa}\Inc{primcount} \InitDet \theprimcount\ \ComplexSqrt. %\qquad%
		\\[.5cm] \par
%		\ifthenelse{\value{tmpa}<#2}{\Inc{tmpa}\Inc{primcount} \InitDet[4] \theprimcount \InvertSystem}{} \\[.5cm] \par%
	 }%
}

\newcommand\SolveComplexSqrt[1]{%
	\ifthenelse{\equal{#1}{O}}{%
		\IfNegate{detac}{\Inverse{detac}}{}%
		\setcounter{detca}{\value{detaa}*\value{detaa} + \value{detab}*\value{detab}}%
		\setcounter{detcc}{2 * \value{detac}}
		$$%
			\ComplexSqrt = \sqrt[\arabic{detcc}]{\arabic{detca}}\left(\cos\frac{\arctg\frac{\arabic{detab}}{\arabic{detaa}}+2\pi k}{\arabic{detac}} + i \sin\frac{\arctg\frac{\arabic{detab}}{\arabic{detaa}}+2\pi k}{\arabic{detac}}\right).%
		$$%
	}{%
	}%
}

\newcommand\MultiSolveComplexSqrt[3][\time]{%
	\reinitrand[seed=#1, first=-9, last=9]%
	\ToNull{tmpa}%
	\whiledo{\value{tmpa}<#2}{%
		\Inc{tmpa}\Inc{primcount}\theprimcount\ \InitDet \SolveComplexSqrt{#3} \par%\\[.5cm] \par%
	 }%
}

% Возведение в степень комплексного числа
\newcommand\ComplexPower{%
	\ensuremath{(\TheComplex{detaa}{detab})^{\arabic{detac}}}%
}

\newcommand\MultiComplexPower[2][\time]{%
	\reinitrand[seed=#1, first=-9, last=9]%
	\ToNull{tmpa}%
	\whiledo{\value{tmpa}<#2}{%
		\Inc{tmpa}\Inc{primcount} \InitDet \theprimcount\ \ComplexPower. %\qquad%
		\\[.5cm] \par
%		\ifthenelse{\value{tmpa}<#2}{\Inc{tmpa}\Inc{primcount} \InitDet[4] \theprimcount \InvertSystem}{} \\[.5cm] \par%
	 }%
}

\newcommand\SolveComplexPower[1]{%
	\ifthenelse{\equal{#1}{O}}{%
		\setcounter{detca}{\value{detaa}*\value{detaa} + \value{detab}*\value{detab}}%
		\setcounter{detcc}{\value{detac} / 2}
		$$%
			\ComplexPower = \ifthenelse{\isodd{\value{detac}}}{\arabic{detca}^{\frac{\arabic{detac}}{2}}}{\arabic{detca}^{\arabic{detcc}}}\left(\cos\left(\arabic{detac}\arctg\frac{\arabic{detab}}{\arabic{detaa}}\right) + i \sin\left(\arabic{detac}\arctg\frac{\arabic{detab}}{\arabic{detaa}}\right)\right).%
		$$%
	}{%
	}%
}

\newcommand\MultiSolveComplexPower[3][\time]{%
	\reinitrand[seed=#1, first=-9, last=9]%
	\ToNull{tmpa}%
	\whiledo{\value{tmpa}<#2}{%
		\Inc{tmpa}\Inc{primcount}\theprimcount\ \InitDet \SolveComplexPower{#3} \par%\\[.5cm] \par%
	 }%
}

% Возведение одного комплексного числа в другое
\newcommand\ComplexXy{%
	\ensuremath{(\arabic{detab} i)^{(\arabic{detaa} i)}}%
}

\newcommand\MultiComplexXy[2][\time]{%
	\reinitrand[seed=#1, first=-9, last=9]%
	\ToNull{tmpa}%
	\whiledo{\value{tmpa}<#2}{%
		\Inc{tmpa}\Inc{primcount} \InitDet \theprimcount\ \ComplexXy. %\qquad%
		\\[.5cm] \par
%		\ifthenelse{\value{tmpa}<#2}{\Inc{tmpa}\Inc{primcount} \InitDet[4] \theprimcount \InvertSystem}{} \\[.5cm] \par%
	 }%
}

\newcommand\SolveComplexXy[1]{%
	\ifthenelse{\equal{#1}{O}}{%
		\setcounter{detcc}{\value{detaa} / 2}%
		$$%
			\ComplexXy = e^{\IfNegate{detaa}{}{-}\ifthenelse{\isodd{\value{detaa}}}{\frac{\arabic{detaa}\pi}{2}}{\arabic{detcc}\pi}}\left(\cos(\arabic{detaa}\ln\arabic{detab}) + i \sin(\arabic{detaa}\ln\arabic{detab})\right).%
		$$%
	}{%
	}%
}

\newcommand\MultiSolveComplexXy[3][\time]{%
	\reinitrand[seed=#1, first=-9, last=9]%
	\ToNull{tmpa}%
	\whiledo{\value{tmpa}<#2}{%
		\Inc{tmpa}\Inc{primcount}\theprimcount\ \InitDet \SolveComplexXy{#3} \par%\\[.5cm] \par%
	 }%
}


\newcommand\BasisFour[1]{%
	$#1_1 = (\thedetaa, \thedetba, \thedetca, \thedetda)^T$;\ 
	$#1_2 = (\thedetab, \thedetbb, \thedetcb, \thedetdb)^T$;\ 
	$#1_3 = (\thedetac, \thedetbc, \thedetcc, \thedetdc)^T$;
}

\newcommand\MultiBasisFour[2][\time]{%
	\reinitrand[seed=#1, first=-3, last=3]%
	\ToNull{tmpa}%
	\whiledo{\value{tmpa}<#2}{%
		\Inc{tmpa}\Inc{primcount}\InitDet[4]\theprimcount\ \BasisFour{a}\par%
		\InitDet[4]\centerline{\BasisFour{b}.}\BigPar%
	}%
}

\newcommand\SolveBasisFour[2]{%
	\BasisFour{#1}%
}

\newcommand\MultiSolveBasisFour[3][\time]{%
	\reinitrand[seed=#1, first=-3, last=3]%
	\ToNull{tmpa}%
	\whiledo{\value{tmpa}<#2}{%
		\Inc{tmpa}\Inc{primcount}\InitDet[4]\theprimcount\ \SolveBasisFour{a}{#3}\par% 
		\InitDet[4]\centerline{\SolveBasisFour{b}{#3}.}\ToDoAns\BigPar% 
	}%
}

\newcommand\BasisFive[1]{%
	$#1_1 = (\thedetaa, \thedetba, \thedetca)^T$;\ %
	$#1_2 = (\thedetab, \thedetbb, \thedetcb)^T$;\ % 
	$#1_3 = (\thedetac, \thedetbc, \thedetcc)^T$; %
}

\newcommand\MultiBasisFive[2][\time]{%
	\reinitrand[seed=#1, first=-3, last=3]%
	\ToNull{tmpa}%
	\whiledo{\value{tmpa}<#2}{%
		\Inc{tmpa}\Inc{primcount}\InitDet\theprimcount\ \BasisFive{a}\par%
		\InitDet\centerline{\BasisFive{b}.}\BigPar%
	}%
}

\newcommand\SolveBasisFive[2]{%
	\BasisFive{#1}%
}

\newcommand\MultiSolveBasisFive[3][\time]{%
	\reinitrand[seed=#1, first=-3, last=3]%
	\ToNull{tmpa}%
	\whiledo{\value{tmpa}<#2}{%
		\Inc{tmpa}\Inc{primcount}\InitDet\theprimcount\ \SolveBasisFive{a}{#3}\par%
		\InitDet\centerline{\SolveBasisFive{b}{#3}.}\ToDoAns\BigPar%
	}%
}

\newcommand\BasisSix{%
	$%
		F = \begin{pmatrix} \thedetaa & \thedetab \\ \thedetac & \thedetad \end{pmatrix}, \qquad% 
		G = \begin{pmatrix} \thedetba & \thedetbb \\ \thedetbc & \thedetbd \end{pmatrix}, \qquad% 
		K = \begin{pmatrix} \thedetca & \thedetcb \\ \thedetcc & \thedetcd \end{pmatrix}%
	$, \par% 
	\centerline{$%		
		M = \begin{pmatrix} \thedetda & \thedetdb \\ \thedetdc & \thedetdd \end{pmatrix}, \qquad%
		A = \begin{pmatrix} \thedetea & \thedeteb \\ \thedetec & \thedeted \end{pmatrix}
	$}%
}

\newcommand\MultiBasisSix[2][\time]{%
	\reinitrand[seed=#1, first=-3, last=3]%
	\ToNull{tmpa}%
	\whiledo{\value{tmpa}<#2}{%
		\Inc{tmpa}\Inc{primcount}\InitDet[5]\theprimcount\ \BasisSix\BigPar%
	}%
}

\newcommand\SolveBasisSix[1]{%
	\BasisSix%
}

\newcommand\MultiSolveBasisSix[3][\time]{%
	\reinitrand[seed=#1, first=-3, last=3]%
	\ToNull{tmpa}%
	\whiledo{\value{tmpa}<#2}{%
		\Inc{tmpa}\Inc{primcount}\InitDet[5]\theprimcount\ \SolveBasisSix{#3}\ToDoAns\BigPar%
	}%
}

\newcommand\BasisSeven[1]{%
	$#1_1 = (\thedetaa, \thedetba, \thedetca)$;\ %
	$#1_2 = (\thedetab, \thedetbb, \thedetcb)$;\ % 
	$#1_3 = (\thedetac, \thedetbc, \thedetcc)$; %
}

\newcommand\MultiBasisSeven[2][\time]{%
	\reinitrand[seed=#1, first=-3, last=3]%
	\ToNull{tmpa}%
	\whiledo{\value{tmpa}<#2}{%
		\Inc{tmpa}\Inc{primcount}\InitDet\theprimcount\ \BasisSeven{e}\par%
		\InitDet\centerline{\BasisSeven{e'}.}\BigPar%
	}%
}

\newcommand\SolveBasisSeven[2]{%
	\BasisSeven{#1}%
}

\newcommand\MultiSolveBasisSeven[3][\time]{%
	\reinitrand[seed=#1, first=-3, last=3]%
	\ToNull{tmpa}%
	\whiledo{\value{tmpa}<#2}{%
		\Inc{tmpa}\Inc{primcount}\InitDet\theprimcount\ \SolveBasisSeven{e}{#3}\par%
		\InitDet\centerline{\SolveBasisSeven{e'}{#3}.}\ToDoAns\BigPar%
	}%
}

\newcommand\BasisEight{%
	$%
		a_1 = (\thedetaa, \thedetba, \thedetca, \thedetda, \thedetea);\quad %
		a_2 = (\thedetab, \thedetbb, \thedetcb, \thedetdb, \thedeteb);
	$\par%
	\centerline{$%
		a_3 = (\thedetac, \thedetbc, \thedetcc, \thedetdc, \thedetec);\quad %
		a_4 = (\thedetad, \thedetbd, \thedetcd, \thedetdd, \thedeted);%
	$}%
}

\newcommand\MultiBasisEight[2][\time]{%
	\reinitrand[seed=#1, first=-3, last=3]%
	\ToNull{tmpa}%
	\whiledo{\value{tmpa}<#2}{%
		\Inc{tmpa}\Inc{primcount}\InitDet[5]\theprimcount\ \BasisEight\ \BigPar%
	}%
}

\newcommand\SolveBasisEight[1]{%
	\BasisEight%
}

\newcommand\MultiSolveBasisEight[3][\time]{%
	\reinitrand[seed=#1, first=-3, last=3]%
	\ToNull{tmpa}%
	\whiledo{\value{tmpa}<#2}{%
		\Inc{tmpa}\Inc{primcount}\InitDet[5]\theprimcount\ \SolveBasisEight{#3} \ToDoAns\BigPar%
	}%
}

\newcommand\BasisNine{%
	$%
		U = h(\thedetaa, \thedetab, \thedetac, \thedetad);%
		(\thedetba, \thedetbb, \thedetbc, \thedetbd);%
	$\par%
	\centerline{$%		
		V = h(\thedetca, \thedetcb, \thedetcc, \thedetcd);%
		(\thedetda, \thedetdb, \thedetdc, \thedetdd);\qquad %
		w = (\thedetea, \thedeteb, \thedetec, \thedeted)%
	$}%
}

\newcommand\MultiBasisNine[2][\time]{%
	\reinitrand[seed=#1, first=-3, last=3]%
	\ToNull{tmpa}%
	\whiledo{\value{tmpa}<#2}{%
		\Inc{tmpa}\Inc{primcount}\InitDet[5]\theprimcount\ \BasisNine \BigPar%
	}%
}

\newcommand\SolveBasisNine[1]{%
	\BasisNine%
}

\newcommand\MultiSolveBasisNine[3][\time]{%
	\reinitrand[seed=#1, first=-3, last=3]%
	\ToNull{tmpa}%
	\whiledo{\value{tmpa}<#2}{%
		\Inc{tmpa}\Inc{primcount}\InitDet[5]\theprimcount\ \SolveBasisNine{#3} \ToDoAns\BigPar%
	}%
}

\newcommand\LinearOneA{%
	\ensuremath{% 
		f_1 = (\thedetaa e_1) + (\thedetab e_2);\ 
		f_2 = (\thedetac e_1) + (\thedetad e_2); 
	}
	$$
		A = \begin{pmatrix}
			\thedetba & \thedetbb \\
			\thedetbc & \thedetbd 
		\end{pmatrix}.
	$$%
}

\newcommand\MultiLinearOneA[2][\time]{%
	\reinitrand[seed=#1, first=-5, last=5]%
	\ToNull{tmpa}%
	\whiledo{\value{tmpa}<#2}{%
		\Inc{tmpa}\Inc{primcount}\InitDet[4]\theprimcount\ \LinearOneA \BigPar%\qquad%
	 }%
}

\newcommand\LinearOneB{%
	\ensuremath{% 
		f_1 = (\thedetaa e_1) + (\thedetab e_2) + (\thedetac e_3);\ 
		f_2 = (\thedetad e_1) + (\thedetae e_2) + (\thedetba e_3);\\ 
		f_3 = (\thedetbb e_1) + (\thedetbc e_2) + (\thedetbd e_3);
	}%
	$$
		A = \begin{pmatrix}
			\thedetca & \thedetcb & \thedetcc \\
			\thedetcd & \thedetce & \thedetda \\
			\thedetdb & \thedetdc & \thedetdd
		\end{pmatrix}.
	$$%
}

\newcommand\MultiLinearOneB[2][\time]{%
	\reinitrand[seed=#1, first=-5, last=5]%
	\ToNull{tmpa}%
	\whiledo{\value{tmpa}<#2}{%
		\Inc{tmpa}\Inc{primcount}\InitDet[5]\theprimcount\ \LinearOneB \BigPar%\qquad%
	 }%
}

\newcommand\LinearTwoA{%
	\ensuremath{% 
		a_1 = (\thedetaa, \thedetab);\ 
		a_2 = (\thedetac, \thedetad); 
	}%
	$$
		A = \begin{pmatrix}
			\thedetba & \thedetbb\\
			\thedetbc & \thedetbd
		\end{pmatrix}.
	$$%
	\ensuremath{% 
		b_1 = (\thedetca, \thedetcb);\ 
		b_2 = (\thedetcd, \thedetdd); 
	}%
}

\newcommand\MultiLinearTwoA[2][\time]{%
	\reinitrand[seed=#1, first=-5, last=5]%
	\ToNull{tmpa}%
	\whiledo{\value{tmpa}<#2}{%
		\Inc{tmpa}\Inc{primcount}\InitDet[4]\theprimcount\ \LinearTwoA\BigPar%\qquad%
	 }%
}

\newcommand\LinearTwoB{%
	\ensuremath{% 
		a_1 = (\thedetaa, \thedetab, \thedetac);\ 
		a_2 = (\thedetad, \thedetae, \thedetaf);\ 
		a_3 = (\thedetba, \thedetbb, \thedetbc);
	}%
	$$
		A = \begin{pmatrix}
			\thedetbd & \thedetbe & \thedetbf \\
			\thedetca & \thedetcb & \thedetcc \\
			\thedetcd & \thedetce & \thedetcf
		\end{pmatrix}.
	$$%
	\ensuremath{% 
		b_1 = (\thedetda, \thedetdb, \thedetdc);\ 
		b_2 = (\thedetdd, \thedetde, \thedetdf);\ 
		b_3 = (\thedetea, \thedeteb, \thedetec);
	}%
}

\newcommand\MultiLinearTwoB[2][\time]{%
	\reinitrand[seed=#1, first=-5, last=5]%
	\ToNull{tmpa}%
	\whiledo{\value{tmpa}<#2}{%
		\Inc{tmpa}\Inc{primcount}\InitDet[6]\theprimcount\ \LinearTwoB\BigPar%\qquad%
	 }%
}

\newcommand\LinearThreeA{%
	\ensuremath{% 
		a_1 = (\thedetaa, \thedetab);\ 
		a_2 = (\thedetac, \thedetad);
	}\\%
	\ensuremath{% 
		b_1 = (\thedetba, \thedetbb);\ 
		b_2 = (\thedetbc, \thedetbd);
	}%
}

\newcommand\MultiLinearThreeA[2][\time]{%
	\reinitrand[seed=#1, first=-5, last=5]%
	\ToNull{tmpa}%
	\whiledo{\value{tmpa}<#2}{%
		\Inc{tmpa}\Inc{primcount}\InitDet[4]\theprimcount\ \LinearThreeA\BigPar%\qquad%
	 }%
}

\newcommand\LinearThreeB{%
	\ensuremath{% 
		a_1 = (\thedetaa, \thedetab, \thedetac);\ 
		a_2 = (\thedetad, \thedetae, \thedetaf);\ 
		a_3 = (\thedetba, \thedetbb, \thedetbc);
	}\\ %
	\ensuremath{% 
		b_1 = (\thedetbd, \thedetbe, \thedetbf);\ 
		b_2 = (\thedetca, \thedetcb, \thedetcc);\ 
		b_3 = (\thedetcd, \thedetce, \thedetcf);
	}%
}

\newcommand\MultiLinearThreeB[2][\time]{%
	\reinitrand[seed=#1, first=-5, last=5]%
	\ToNull{tmpa}%
	\whiledo{\value{tmpa}<#2}{%
		\Inc{tmpa}\Inc{primcount}\InitDet[6]\theprimcount\ \LinearThreeB\BigPar%\qquad%
	 }%
}

\newcommand\LinearFourA{%
	$$%
		A = \begin{pmatrix}
			\thedetaa & \thedetab & \thedetac \\
			\thedetba & \thedetbb & \thedetbc \\
			\thedetca & \thedetcb & \thedetcc 
		\end{pmatrix}.
	$$%
}

\newcommand\MultiLinearFourA[2][\time]{%
	\reinitrand[seed=#1, first=-5, last=5]%
	\ToNull{tmpa}%
	\whiledo{\value{tmpa}<#2}{%
		\Inc{tmpa}\Inc{primcount}\InitDet\theprimcount\ \LinearFourA\BigPar%\qquad%
	 }%
}

\newcommand\LinearFourB{%
	$$
		A = \begin{pmatrix}
			\thedetaa & \thedetab & \thedetac & \thedetad\\
			\thedetba & \thedetbb & \thedetbc & \thedetbd\\
			\thedetca & \thedetcb & \thedetcc & \thedetcd\\
			\thedetda & \thedetdb & \thedetdc & \thedetdd
		\end{pmatrix}.
	$$%
}

\newcommand\MultiLinearFourB[2][\time]{%
	\reinitrand[seed=#1, first=-5, last=5]%
	\ToNull{tmpa}%
	\whiledo{\value{tmpa}<#2}{%
		\Inc{tmpa}\Inc{primcount}\InitDet[4]\theprimcount\ \LinearFourB\BigPar%\qquad%
	 }%
}

\newcommand\LinearFiveA{%
	$$
		A = \begin{pmatrix}
			\thedetaa & \thedetab & \thedetac & \thedetad\\
			\thedetba & \thedetbb & \thedetbc & \thedetbd\\
			\thedetca & \thedetcb & \thedetcc & \thedetcd
		\end{pmatrix}.
	$$%
}

\newcommand\MultiLinearFiveA[2][\time]{%
	\reinitrand[seed=#1, first=-5, last=5]%
	\ToNull{tmpa}%
	\whiledo{\value{tmpa}<#2}{%
		\Inc{tmpa}\Inc{primcount}\InitDet[4]\theprimcount\ \LinearFiveA\BigPar%\qquad%
	 }%
}

\newcommand\LinearFiveB{%
	$$
		A = \begin{pmatrix}
			\thedetaa & \thedetab & \thedetac & \thedetad & \thedetae\\
			\thedetba & \thedetbb & \thedetbc & \thedetbd & \thedetab\\
			\thedetca & \thedetcb & \thedetcc & \thedetcd & \thedetac\\
			\thedetda & \thedetdb & \thedetdc & \thedetdd & \thedetad\\
			\thedetea & \thedeteb & \thedetec & \thedeted & \thedetee
		\end{pmatrix}.
	$$%
}

\newcommand\MultiLinearFiveB[2][\time]{%
	\reinitrand[seed=#1, first=-5, last=5]%
	\ToNull{tmpa}%
	\whiledo{\value{tmpa}<#2}{%
		\Inc{tmpa}\Inc{primcount}\InitDet[5]\theprimcount\ \LinearFiveB\BigPar%\qquad%
	 }%
}

\newcommand\LinearSixA{%
	\ensuremath{%
		a = (\thedetaa, \thedetba, \thedetca); \qquad%
		b = (\thedetab, \thedetbb, \thedetcb).% 
	}%
}

\newcommand\MultiLinearSixA[2][\time]{%
	\reinitrand[seed=#1, first=-5, last=5]%
	\ToNull{tmpa}%
	\whiledo{\value{tmpa}<#2}{%
		\Inc{tmpa}\Inc{primcount}\InitDet\theprimcount\ \LinearSixA\BigPar%\qquad%
	 }%
}

\newcommand\LinearSixB{%
	\ensuremath{%
		a = (\thedetaa, \thedetab, \thedetac, \thedetad, \thedetae); \qquad%
		b = (\thedetba, \thedetbb, \thedetbc, \thedetbd, \thedetbe).% 
	}%
}

\newcommand\MultiLinearSixB[2][\time]{%
	\reinitrand[seed=#1, first=-5, last=5]%
	\ToNull{tmpa}%
	\whiledo{\value{tmpa}<#2}{%
		\Inc{tmpa}\Inc{primcount}\InitDet[5]\theprimcount\ \LinearSixB\BigPar%\qquad%
	 }%
}

\newcommand\LinearSevenA{%
	\ensuremath{%
		a_1 = (\thedetaa, \thedetab, \thedetac); \qquad%
		a_2 = (\thedetba, \thedetbb, \thedetbc).% 
	}%
}

\newcommand\MultiLinearSevenA[2][\time]{%
	\reinitrand[seed=#1, first=-5, last=5]%
	\ToNull{tmpa}%
	\whiledo{\value{tmpa}<#2}{%
		\Inc{tmpa}\Inc{primcount}\InitDet\theprimcount\ \LinearSevenA\BigPar%\qquad%
	 }%
}

\newcommand\LinearSevenB{%
	\ensuremath{%
		a_1 = (\thedetaa, \thedetab, \thedetac, \thedetad, \thedetae); \qquad%
		a_2 = (\thedetba, \thedetbb, \thedetbc, \thedetbd, \thedetbe);% 
	}%
}

\newcommand\MultiLinearSevenB[2][\time]{%
	\reinitrand[seed=#1, first=-5, last=5]%
	\ToNull{tmpa}%
	\whiledo{\value{tmpa}<#2}{%
		\Inc{tmpa}\Inc{primcount}\InitDet[5]\theprimcount\ \LinearSevenB\BigPar%\qquad%
	 }%
}

\newcommand\LinearEight{%
	\ensuremath{% 
		f_1 = (\thedetaa, \thedetab, \thedetac)^T;\ % 
		f_2 = (\thedetad, \thedetae, \thedetaf)^T;\ % 
		f_3 = (\thedetba, \thedetbb, \thedetbc)^T;%
	}%
	$$%
		A = \begin{pmatrix}%
			\thedetbd & \thedetbe & \thedetbf \\%
			\thedetca & \thedetcb & \thedetcc \\%
			\thedetcd & \thedetce & \thedetcf%
		\end{pmatrix}; \qquad%
		B = \begin{pmatrix}%
			\thedetda & \thedetdb & \thedetdc \\%
			\thedetdd & \thedetde & \thedetdf \\%
			\thedetea & \thedeteb & \thedetec%
		\end{pmatrix}; \qquad% 
		x = \begin{pmatrix}%
			\thedeted \\ \thedetee \\ \thedetef%
		\end{pmatrix}.%
	$$%
}

\newcommand\MultiLinearEight[1][\time]{%
	\reinitrand[seed=#1, first=-5, last=5]%
	\ToNull{tmpa}%
	\Inc{tmpa} \textbf{\alph{tmpa}). }%
	Проверить, что $f = (f_1; f_2; f_3)$ базис в $R^3$ и найти матрицы перехода $Ce-f$ , $Cf-e$.
	\smallskip \par
	\Inc{tmpa} \textbf{\alph{tmpa}). }%
	Определите координаты вектора $y = A \circ B(x)$ в базисе $f$.
	\smallskip \par
	\Inc{tmpa} \textbf{\alph{tmpa}). }%
	Найти матрицы оператора $A^{-1}$ в базисах $e$ и $f$.
	\smallskip \par
	\Inc{tmpa} \textbf{\alph{tmpa}). }%
	Найти размерность ядра и образа оператора $A$.
	\smallskip \par
	\Inc{tmpa} \textbf{\alph{tmpa}). }%
	Найти размерность ядра и образа оператора $B$.
	\smallskip \par
	\Inc{tmpa} \textbf{\alph{tmpa}). }%
	Постройте ортонормированный базис ядра и образа оператора $A$.
	\smallskip \par
	\Inc{tmpa} \textbf{\alph{tmpa}). }%
	Постройте ортонормированный базис ядра и образа оператора $B$.
	\smallskip \par
	\Inc{tmpa} \textbf{\alph{tmpa}). }%
	Найдите собственные числа и собственные вектора операторов $A$.
	\smallskip \par
	\Inc{tmpa} \textbf{\alph{tmpa}). }%
	Найдите собственные числа и собственные вектора операторов $B$.
	\smallskip \par
	\Inc{tmpa} \textbf{\alph{tmpa}). }%
	Выпишите матрицы операторов $A$ и $B$ в базисе из собственных векторов.
	\smallskip \par
	\Inc{primcount}\InitDet[6]\theprimcount\ \LinearEight\BigPar%\qquad%
	\Inc{primcount}\InitDet[6]\theprimcount\ \LinearEight\BigPar%\qquad%
}

\newcommand\JordanOne{%
	$$%
		A = \left(\PrintMatrix{3}\right).%
	$$%
}

\newcommand\MultiJordanOne[2][\time]{%
	\reinitrand[seed=#1, first=-5, last=5]%
	\ToNull{tmpa}%
	\whiledo{\value{tmpa}<#2}{%
		\Inc{tmpa}\Inc{primcount}\InitDet\theprimcount\ \JordanOne\BigPar%
	}%
}

\newcommand\SolveJordanOne[1]{%
	\JordanOne%
	\ToDoAns%
}

\newcommand\MultiSolveJordanOne[3][\time]{%
	\reinitrand[seed=#1, first=-5, last=5]%
	\ToNull{tmpa}%
	\whiledo{\value{tmpa}<#2}{%
		\Inc{tmpa}\Inc{primcount}\InitDet\theprimcount\ \SolveJordanOne{#3}\BigPar%
	}%
}

\newcommand\JordanTwo{%
	$$%
		A = \left(\PrintMatrix{3}\right).%
	$$%
}

\newcommand\MultiJordanTwo[2][\time]{%
	\reinitrand[seed=#1, first=-5, last=5]%
	\ToNull{tmpa}%
	\whiledo{\value{tmpa}<#2}{%
		\Inc{tmpa}\Inc{primcount}\InitDet\theprimcount\ \JordanTwo\BigPar%
	 }%
}

\newcommand\SolveJordanTwo[1]{%
	\JordanTwo%
	\ToDoAns%
}

\newcommand\MultiSolveJordanTwo[3][\time]{%
	\reinitrand[seed=#1, first=-5, last=5]%
	\ToNull{tmpa}%
	\whiledo{\value{tmpa}<#2}{%
		\Inc{tmpa}\Inc{primcount}\InitDet\theprimcount\ \SolveJordanTwo{#3}\BigPar%
	}%
}

\newcommand\JordanThree{%
	$$%
		A = \left(\PrintMatrix{4}\right).%
	$$%
}

\newcommand\MultiJordanThree[2][\time]{%
	\reinitrand[seed=#1, first=-5, last=5]%
	\ToNull{tmpa}%
	\whiledo{\value{tmpa}<#2}{%
		\Inc{tmpa}\Inc{primcount}\InitDet[4]\theprimcount\ \JordanThree\BigPar%
	 }%
}

\newcommand\SolveJordanThree[1]{%
	\JordanThree%
	\ToDoAns%
}

\newcommand\MultiSolveJordanThree[3][\time]{%
	\reinitrand[seed=#1, first=-5, last=5]%
	\ToNull{tmpa}%
	\whiledo{\value{tmpa}<#2}{%
		\Inc{tmpa}\Inc{primcount}\InitDet[4]\theprimcount\ \SolveJordanThree{#3}\BigPar%
	}%
}

\newcommand\JordanFour{%
	$$
		A = \begin{pmatrix}
			1 & \thedetaa & \thedetab & \thedetac & \thedetad \\
			0 & 1 & \thedetbb & \thedetbc & \thedetbd\\
			0 & 0 & 1 & \thedetcc & \thedetcd\\
			0 & 0 & 0 & 1 & \thedetdd\\
			0 & 0 & 0 & 0 & 1
		\end{pmatrix}.
	$$%
}

\newcommand\MultiJordanFour[2][\time]{%
	\reinitrand[seed=#1, first=-5, last=5]%
	\ToNull{tmpa}%
	\whiledo{\value{tmpa}<#2}{%
		\Inc{tmpa}\Inc{primcount}\InitDet[4]\theprimcount\ \JordanFour\BigPar%
	 }%
}

\newcommand\SolveJordanFour[1]{%
	\JordanFour%
	\ToDoAns%
}

\newcommand\MultiSolveJordanFour[3][\time]{%
	\reinitrand[seed=#1, first=-5, last=5]%
	\ToNull{tmpa}%
	\whiledo{\value{tmpa}<#2}{%
		\Inc{tmpa}\Inc{primcount}\InitDet[4]\theprimcount\ \SolveJordanFour{#3}\BigPar%
	}%
}

\newcommand\JordanFive{%
	$$
		A = \begin{pmatrix}
			1 & \thedetaa & \thedetab & \thedetac & \thedetad & \thedetae\\
			0 & 1 & \thedetbb & \thedetbc & \thedetbd & \thedetbe\\
			0 & 0 & 1 & \thedetcc & \thedetcd & \thedetce\\
			0 & 0 & 0 & 1 & \thedetdd & \thedetde\\
			0 & 0 & 0 & 0 & 1 & \thedetee \\
			0 & 0 & 0 & 0 & 0 & 1
		\end{pmatrix}.
	$$%
}

\newcommand\MultiJordanFive[2][\time]{%
	\reinitrand[seed=#1, first=-5, last=5]%
	\ToNull{tmpa}%
	\whiledo{\value{tmpa}<#2}{%
		\Inc{tmpa}\Inc{primcount}\InitDetFive\theprimcount\ \JordanFive\BigPar%
	 }%
}

\newcommand\SolveJordanFive[1]{%
	\JordanFive%
	\ToDoAns%
}

\newcommand\MultiSolveJordanFive[3][\time]{%
	\reinitrand[seed=#1, first=-5, last=5]%
	\ToNull{tmpa}%
	\whiledo{\value{tmpa}<#2}{%
		\Inc{tmpa}\Inc{primcount}\InitDetFive\theprimcount\ \SolveJordanFive{#3}\BigPar%
	}%
}

\newcommand\JordanSix{%
	\ToNull{tmpc}%
	\IfNegate{detaa}{\Imm{tmpc}{detaa}}{\Ipp{tmpc}{detaa}}%
	\IfNegate{detab}{\Imm{tmpc}{detab}}{\Ipp{tmpc}{detab}}%
	\IfNegate{detac}{\Imm{tmpc}{detac}}{\Ipp{tmpc}{detac}}%
	\IfNegate{detba}{\Imm{tmpc}{detba}}{\Ipp{tmpc}{detba}}%
	\IfNegate{detbb}{\Imm{tmpc}{detbb}}{\Ipp{tmpc}{detbb}}%
	\IfNegate{detbc}{\Imm{tmpc}{detbc}}{\Ipp{tmpc}{detbc}}%
	\IfNegate{detca}{\Imm{tmpc}{detca}}{\Ipp{tmpc}{detca}}%
	\IfNegate{detcb}{\Imm{tmpc}{detcb}}{\Ipp{tmpc}{detcb}}%
	\IfNegate{detcc}{\Imm{tmpc}{detcc}}{\Ipp{tmpc}{detcc}}%
	$$%
		n = \arabic{tmpc}; \qquad A = \left(\PrintMatrix{2}\right).%
	$$%
}

\newcommand\MultiJordanSix[2][\time]{%
	\reinitrand[seed=#1, first=-5, last=5]%
	\ToNull{tmpa}%
	\whiledo{\value{tmpa}<#2}{%
		\Inc{tmpa}\Inc{primcount}\InitDet\theprimcount\ \JordanSix\BigPar%
	 }%
}

\newcommand\SolveJordanSix[1]{%
	\JordanSix%
	\ToDoAns%
}

\newcommand\MultiSolveJordanSix[3][\time]{%
	\reinitrand[seed=#1, first=-5, last=5]%
	\ToNull{tmpa}%
	\whiledo{\value{tmpa}<#2}{%
		\Inc{tmpa}\Inc{primcount}\InitDet\theprimcount\ \SolveJordanSix{#3}\BigPar%
	}%
}

\newcommand\JordanSeven{%
	$$%
		A = \left(\PrintMatrix{2}\right).%
	$$%
}

\newcommand\MultiJordanSeven[2][\time]{%
	\reinitrand[seed=#1, first=-5, last=5]%
	\ToNull{tmpa}%
	\whiledo{\value{tmpa}<#2}{%
		\Inc{tmpa}\Inc{primcount}\InitDet\theprimcount\ \JordanSeven\BigPar%
	 }%
}

\newcommand\SolveJordanSeven[1]{%
	\JordanSeven%
	\ToDoAns%
}

\newcommand\MultiSolveJordanSeven[3][\time]{%
	\reinitrand[seed=#1, first=-5, last=5]%
	\ToNull{tmpa}%
	\whiledo{\value{tmpa}<#2}{%
		\Inc{tmpa}\Inc{primcount}\InitDet\theprimcount\ \SolveJordanSeven{#3}\BigPar%
	}%
}

\newcommand\JordanEight{%
	$$%
		A = \left(\PrintMatrix{2}\right).%
	$$%
}

\newcommand\MultiJordanEight[2][\time]{%
	\reinitrand[seed=#1, first=-5, last=5]%
	\ToNull{tmpa}%
	\whiledo{\value{tmpa}<#2}{%
		\Inc{tmpa}\Inc{primcount}\InitDet\theprimcount\ \JordanEight\BigPar%
	 }%
}

\newcommand\SolveJordanEight[1]{%
	\JordanEight%
	\ToDoAns%
}

\newcommand\MultiSolveJordanEight[3][\time]{%
	\reinitrand[seed=#1, first=-5, last=5]%
	\ToNull{tmpa}%
	\whiledo{\value{tmpa}<#2}{%
		\Inc{tmpa}\Inc{primcount}\InitDet\theprimcount\ \SolveJordanEight{#3}\BigPar%
	}%
}

\newcommand\QuadSimple{%
	$%
		Q = \arabic{detab} x_1 x_2 \IfNegate{detac}{}{+} \arabic{detac} x_1 x_3 \IfNegate{detbc}{}{+} \arabic{detbc} x_2 x_3%
	$%
}

\newcommand\MultiQuadSimple[2][\time]{%
	\reinitrand[seed=#1, first=-9, last=9]%
	\ToNull{tmpa}%
	\whiledo{\value{tmpa}<#2}{%
		\Inc{tmpa}\InitDet\Example{\QuadSimple.}{}%
	 }%
}

\newcommand\SolveQuadSimple[1]{%
	\QuadSimple%
	\ToDoAns%
}

\newcommand\MultiSolveQuadSimple[3][\time]{%
	\reinitrand[seed=#1, first=-9, last=9]%
	\ToNull{tmpa}%
	\whiledo{\value{tmpa}<#2}{%
		\Inc{tmpa}\InitDet\Example{\SolveQuadSimple{#3}}{}%
	}%
}


% Набор команд для работы со счётчиками
\newcommand{\Inc}[1]{\addtocounter{#1}{1}}
\newcommand{\Dec}[1]{\addtocounter{#1}{-1}}
\newcommand{\TripleMulti}[4]{\setcounter{#1}{\value{#2}*\value{#3}*\value{#4}}}
\newcommand{\Multi}[3]{\setcounter{#1}{\value{#2}*\value{#3}}}
\newcommand{\ToNull}[1]{\setcounter{#1}{0}}
\newcommand{\Move}[2]{\setcounter{#1}{\value{#2}}}
\newcommand{\Ipp}[2]{\addtocounter{#1}{\value{#2}}}
\newcommand{\Imm}[2]{\addtocounter{#1}{-\value{#2}}}
\newcommand{\Swap}[2]{\Move{temp}{#1}\Move{#1}{#2}\Move{#2}{temp}}
\newcommand{\Inverse}[1]{\setcounter{#1}{-\value{#1}}}

%Проверка на ноль
\newcommand\IfNull[3]{%
	\ifthenelse{\value{#1}=0}{#2}{#3}%
}

%Проверка на отрицательность
\newcommand\IfNegate[3]{%
	\ifthenelse{\value{#1}<0}{#2}{#3}%
}

%Проверка на единицу
\newcommand\IfOne[3]{%
	\ifthenelse{\value{#1}=1}{#2}{#3}%
}

%Проверка на минус единицу
\newcommand\IfNOne[3]{%
	\ifthenelse{\value{#1}=-1}{#2}{#3}%
}


% Счётчики для 3х3
\newcounter{detaa}
\newcounter{detab}
\newcounter{detac}
\newcounter{detba}
\newcounter{detbb}
\newcounter{detbc}
\newcounter{detca}
\newcounter{detcb}
\newcounter{detcc}

% Счётчики для 4х4
\newcounter{detda}
\newcounter{detdb}
\newcounter{detdc}
\newcounter{detdd}
\newcounter{detad}
\newcounter{detbd}
\newcounter{detcd}
\newcounter{torow}
\newcounter{tocol}

% Счётчики для 5х5
\newcounter{detea}
\newcounter{deteb}
\newcounter{detec}
\newcounter{deted}
\newcounter{detee}
\newcounter{detae}
\newcounter{detbe}
\newcounter{detce}
\newcounter{detde}

% Счётчики для 6х6
\newcounter{detfa}
\newcounter{detfb}
\newcounter{detfc}
\newcounter{detfd}
\newcounter{detfe}
\newcounter{detff}
\newcounter{detaf}
\newcounter{detbf}
\newcounter{detcf}
\newcounter{detdf}
\newcounter{detef}

% Счётчики для промежуточных вычислений
\newcounter{dettmpa}
\newcounter{dettmpb}
\newcounter{dettmpc}

% Счётчики для умножения матриц
\newcounter{mataaa}
\newcounter{mataab}
\newcounter{mataac}
\newcounter{mataad}
\newcounter{mataba}
\newcounter{matabb}
\newcounter{matabc}
\newcounter{matabd}
\newcounter{mataca}
\newcounter{matacb}
\newcounter{matacc}
\newcounter{matacd}
\newcounter{matada}
\newcounter{matadb}
\newcounter{matadc}
\newcounter{matadd}

%	----------------------		Определение новых команд 		---------------------

% Инициализация матрицы
\newcommand{\InitDet}[1][3]{%
	\ifcase #1 \of 
	\or %
		\rand\Move{detaa}{rand}%
	\or\InitDet[1]%	
		\rand\Move{detab}{rand}%
		\rand\Move{detba}{rand}%
		\rand\Move{detbb}{rand}%
	\or\InitDet[2]%
		\rand\Move{detac}{rand}%
		\rand\Move{detbc}{rand}%
		\rand\Move{detca}{rand}%
		\rand\Move{detcb}{rand}%
		\rand\Move{detcc}{rand}%
	\or\InitDet[3]%
		\rand\Move{detda}{rand}%
		\rand\Move{detdb}{rand}%
		\rand\Move{detdc}{rand}%
		\rand\Move{detad}{rand}%
		\rand\Move{detbd}{rand}%
		\rand\Move{detcd}{rand}%
		\rand\Move{detdd}{rand}%
	\or\InitDet[4]%
		\rand\Move{detae}{rand}%
		\rand\Move{detbe}{rand}%
		\rand\Move{detce}{rand}%
		\rand\Move{detde}{rand}%
		\rand\Move{detea}{rand}%
		\rand\Move{deteb}{rand}%
		\rand\Move{detec}{rand}%
		\rand\Move{deted}{rand}%
		\rand\Move{detee}{rand}%
	\or\InitDet[5]%
		\rand\Move{detaf}{rand}%
		\rand\Move{detbf}{rand}%
		\rand\Move{detcf}{rand}%
		\rand\Move{detdf}{rand}%
		\rand\Move{detef}{rand}%
		\rand\Move{detff}{rand}%
		\rand\Move{detfa}{rand}%
		\rand\Move{detfb}{rand}%
		\rand\Move{detfc}{rand}%
		\rand\Move{detfd}{rand}%
		\rand\Move{detfe}{rand}%
	\fi%
}

\newcommand{\InitDetFive}{%
	\rand\Move{detaa}{rand}%
	\rand\Move{detab}{rand}%
	\rand\Move{detac}{rand}%
	\rand\Move{detad}{rand}%
	\rand\Move{detae}{rand}%
	\rand\Move{detba}{rand}%
	\rand\Move{detbb}{rand}%
	\rand\Move{detbc}{rand}%
	\rand\Move{detbd}{rand}%
	\rand\Move{detbe}{rand}%
	\rand\Move{detca}{rand}%
	\rand\Move{detcb}{rand}%
	\rand\Move{detcc}{rand}%
	\rand\Move{detcd}{rand}%
	\rand\Move{detce}{rand}%
	\rand\Move{detda}{rand}%
	\rand\Move{detdb}{rand}%
	\rand\Move{detdc}{rand}%
	\rand\Move{detdd}{rand}%
	\rand\Move{detde}{rand}%
	\rand\Move{detea}{rand}%
	\rand\Move{deteb}{rand}%
	\rand\Move{detec}{rand}%
	\rand\Move{deted}{rand}%
	\rand\Move{detee}{rand}%
}

% Вывод на экран определителя 2х2
\newcommand{\DetTwo}{\ensuremath{\D \left| \begin{array}{rr} \arabic{detaa} & \arabic{detab} \\ \arabic{detba} & \arabic{detbb} \end{array} \right|}}

% Вывод заданного количества определителей 2х2. Можно указать число, для генерации конкретных вариантов.
\newcommand{\MultiDetTwo}[2][\time]{%
	\reinitrand[seed=#1, first=-9, last=9]%
	\ToNull{tmpa}%
	\whiledo{\value{tmpa}<#2}{%
		\Inc{tmpa}\Example{\InitDet\DetTwo.}{}%
%		\ifthenelse{\value{tmpa}<#2}{\Inc{tmpa}\Example{\InitDet\DetTwo.}{}}{}%
	 }%
}

% Вывод решения определителя 3х3
\newcommand{\SolveDetTwo}[2][0]{%	
	\ToNull{dettmpb}%
	\ensuremath{%
		\DetTwo \ifthenelse{\equal{#2}{S}}{ = %
			\ifthenelse{\equal{#1}{0}}{}{-1 \cdot(}%
			\alg{detaa} \cdot \alg{detbb} - \alg{detab} \cdot \alg{detba}%
			\ifthenelse{\equal{#1}{0}}{=}{) = -1 \cdot(}
		}{}%
		\Multi{dettmpa}{detaa}{detbb}\Ipp{dettmpb}{dettmpa}\ifthenelse{\equal{#2}{S}}{\alg{dettmpa} - }{}%
		\Multi{dettmpa}{detab}{detba}\Imm{dettmpb}{dettmpa}\ifthenelse{\equal{#2}{S}}{\alg{dettmpa}}{} %
		= \ifthenelse{\equal{#2}{S}}{\ifthenelse{\equal{#1}{0}}{}{)}}{} %
		\ifthenelse{\equal{#1}{0}}{}{\setcounter{dettmpb}{-\value{dettmpb}}}\thedettmpb.%
	}%
}

% Вывод заданного количества решений определителей 3х3. Можно указать число, для генерации конкретных вариантов.
\newcommand{\MultiSolveDetTwo}[3][\time]{%
	\reinitrand[seed=#1, first=-9, last=9]%
	\ToNull{tmpa}%
	\whiledo{\value{tmpa}<#2}{%
		\Inc{tmpa}\Example{\InitDet\SolveDetTwo{#3}}{}%
%		\ifthenelse{\value{tmpa}<#2}{\Inc{tmpa}\Inc{primcount}\InitDet\theprimcount\ \SolveDetTwo{#3}}{} \BigPar%
	}%
}

% Вывод на экран определителя 3х3
\newcommand{\DetThree}{\ensuremath{\D \left| \PrintMatrix{3} \right|}}

% Вывод заданного количества определителей 3х3. Можно указать число, для генерации конкретных вариантов.
\newcommand{\MultiDetThree}[2][\time]{%
	\reinitrand[seed=#1, first=-9, last=9]%
	\ToNull{tmpa}%
	\whiledo{\value{tmpa}<#2}{%
		\Inc{tmpa}\Example{\InitDet\DetThree.}{}%
%		\ifthenelse{\value{tmpa}<#2}{\Inc{tmpa}\Inc{primcount}\InitDet\theprimcount\ \DetThree.}{} \BigPar%%
	 }%
}

% Вывод решения определителя 3х3
\newcommand{\SolveDetThree}[2][0]{%	
	\ToNull{dettmpb}%
	\ensuremath{%
		\DetThree \ifthenelse{\equal{#2}{S}}{ = %
			\ifthenelse{\equal{#1}{0}}{}{-1 \cdot(}\alg{detaa} \cdot \alg{detbb} \cdot \alg{detcc} + %
			\alg{detab} \cdot \alg{detbc} \cdot \alg{detca} + %
			\alg{detba} \cdot \alg{detcb} \cdot \alg{detac} - %
			\alg{detac} \cdot \alg{detbb} \cdot \alg{detca} - %
			\alg{detaa} \cdot \alg{detbc} \cdot \alg{detcb} - %
			\alg{detab} \cdot \alg{detba} \cdot \alg{detcc} %
			\ifthenelse{\equal{#1}{0}}{=}{) = -1 \cdot(}%
		}{}%
			\TripleMulti{dettmpa}{detaa}{detbb}{detcc}\Ipp{dettmpb}{dettmpa}\ifthenelse{\equal{#2}{S}}{\alg{dettmpa} + }{}%
			\TripleMulti{dettmpa}{detab}{detbc}{detca}\Ipp{dettmpb}{dettmpa}\ifthenelse{\equal{#2}{S}}{\alg{dettmpa} + }{}%
			\TripleMulti{dettmpa}{detba}{detcb}{detac}\Ipp{dettmpb}{dettmpa}\ifthenelse{\equal{#2}{S}}{\alg{dettmpa} - }{}%
			\TripleMulti{dettmpa}{detac}{detbb}{detca}\Imm{dettmpb}{dettmpa}\ifthenelse{\equal{#2}{S}}{\alg{dettmpa} - }{}%
			\TripleMulti{dettmpa}{detaa}{detbc}{detcb}\Imm{dettmpb}{dettmpa}\ifthenelse{\equal{#2}{S}}{\alg{dettmpa} - }{}%
			\TripleMulti{dettmpa}{detab}{detba}{detcc}\Imm{dettmpb}{dettmpa}\ifthenelse{\equal{#2}{S}}{\alg{dettmpa} }{}%
			\ifthenelse{\equal{#2}{S}}{\ifthenelse{\equal{#1}{0}}{}{)}}{} = %
			\ifthenelse{\equal{#1}{0}}{}{\setcounter{dettmpb}{-\value{dettmpb}}}\thedettmpb.%
	}%
}

% Вывод заданного количества решений определителей 3х3. Можно указать число, для генерации конкретных вариантов.
\newcommand{\MultiSolveDetThree}[3][\time]{%
	\reinitrand[seed=#1, first=-9, last=9]%
	\ToNull{tmpa}%
	\whiledo{\value{tmpa}<#2}{%
		\Inc{tmpa}\Example{\InitDet\SolveDetThree{#3}}{}%
%		\ifthenelse{\value{tmpa}<#2}{\Inc{tmpa}\Inc{primcount}\InitDet\theprimcount\ \SolveDetThree{#3}}{} \\BigPar%%
	 }%
}

% Вывод на экран определителя 4х4
\newcommand{\DetFour}{\ensuremath{\D \left| \begin{array}{rrrr} \arabic{detaa} & \arabic{detab} & \arabic{detac} & \arabic{detad} \\ \arabic{detba} & \arabic{detbb} & \arabic{detbc} & \arabic{detbd} \\ \arabic{detca} & \arabic{detcb} & \arabic{detcc} & \arabic{detcd} \\ \arabic{detda} & \arabic{detdb} & \arabic{detdc} & \arabic{detdd} \end{array} \right|}}

% Вывод заданного количества определителей 4х4. Можно указать число, для генерации конкретных вариантов.
\newcommand{\MultiDetFour}[2][\time]{%
	\reinitrand[seed=#1, first=-9, last=9]%
	\ToNull{tmpa}%
	\whiledo{\value{tmpa}<#2}{%
		\Inc{tmpa}\Example{\InitDet[4]\setcounter{detdd}{1}\SwapRow{2}\SwapCol{1}\DetFour.}{}%
%		\ifthenelse{\value{tmpa}<#2}{\Inc{tmpa}\Inc{primcount}\InitDet[4]\setcounter{detdd}{1}\SwapRow{2}\SwapCol{1}\theprimcount\ \DetFour.}{} \BigPar%
	 }%
}

% Обмен какой-либо строки с последней строкой
\newcommand{\SwapRow}[1]{%
	\ifcase #1 \of \or \Swap{detaa}{detda}\Swap{detab}{detdb}\Swap{detac}{detdc}\Swap{detad}{detdd} \or \Swap{detba}{detda}\Swap{detbb}{detdb}\Swap{detbc}{detdc}\Swap{detbd}{detdd} \or \Swap{detca}{detda}\Swap{detcb}{detdb}\Swap{detcc}{detdc}\Swap{detcd}{detdd} \fi%
}

% Обмен какого-либо столбца с последним столбцом
\newcommand{\SwapCol}[1]{%
	\ifcase #1 \of \or \Swap{detaa}{detad}\Swap{detba}{detbd}\Swap{detca}{detcd}\Swap{detda}{detdd} \or \Swap{detab}{detad}\Swap{detbb}{detbd}\Swap{detcb}{detcd}\Swap{detdb}{detdd} \or \Swap{detac}{detad}\Swap{detbc}{detbd}\Swap{detcc}{detcd}\Swap{detdc}{detdd} \fi%
}

% Вывод решения определителя 4х4
\newcommand{\SolveDetFour}[3]{%	
	\ToNull{dettmpb}%
	\ensuremath{%
		\DetFour \D \ifthenelse{\equal{#3}{S}}{\Suggestion{#1}{#2}}{} \ModifyDet{#1}{#2} \ifthenelse{\equal{#3}{S}}{\Longrightarrow \DetFour}{} \ToDefault{#1}{#2} = } %
	\ifthenelse{\equal{#3}{S}}{ \\%
		\ensuremath{% 
	  		 = \ifthenelse{\(\isodd{#1} \and \isodd{#2}\) \or \(\not\isodd{#1} \and \not\isodd{#2}\)}{1 \cdot \SolveDetThree{#3}}{-1 \cdot \SolveDetThree[1]{#3}}  %
		}%
	}{\ensuremath{\vphantom{\ifthenelse{\(\isodd{#1} \and \isodd{#2}\) \or \(\not\isodd{#1} \and \not\isodd{#2}\)}{1 \cdot \SolveDetThree{#3}}{-1 \cdot \SolveDetThree[1]{#3}}}} \thedettmpb.}%
}

% Вывод заданного количества решений определителей 4х4. Можно указать число, для генерации конкретных вариантов.
\newcommand{\MultiSolveDetFour}[3][\time]{%
	\reinitrand[seed=#1, first=-9, last=9]%
	\ToNull{tmpa}%
	\whiledo{\value{tmpa}<#2}{%
		\Inc{tmpa}\Example{\InitDet[4]\setcounter{detdd}{1}\GetNum{torow}{tocol}\SwapRow{\arabic{torow}}\SwapCol{\arabic{tocol}}\SolveDetFour{\arabic{torow}}{\arabic{tocol}}{#3}}{}%
	 }%
}

% Получение индексов переноса 1
\newcommand{\GetNum}[2]{%
	\ifthenelse{\isodd{\value{detac}} \and \isodd{\value{detbb}}}{\setcounter{#1}{1}}{}%
	\ifthenelse{\isodd{\value{detac}} \and \not\isodd{\value{detbb}}}{\setcounter{#1}{2}}{}%
	\ifthenelse{\not\isodd{\value{detac}} \and \isodd{\value{detbb}}}{\setcounter{#1}{3}}{}%
	\ifthenelse{\not\isodd{\value{detac}} \and \not\isodd{\value{detbb}}}{\setcounter{#1}{4}}{}%
	\ifthenelse{\isodd{\value{detbc}} \and \isodd{\value{detab}}}{\setcounter{#2}{1}}{}%
	\ifthenelse{\isodd{\value{detbc}} \and \not\isodd{\value{detab}}}{\setcounter{#2}{2}}{}%
	\ifthenelse{\not\isodd{\value{detbc}} \and \isodd{\value{detab}}}{\setcounter{#2}{3}}{}%
	\ifthenelse{\not\isodd{\value{detbc}} \and \not\isodd{\value{detab}}}{\setcounter{#2}{4}}{}%
%\setcounter{#1}{2}\setcounter{#2}{1}
}

% Предложение по приведению к 3х3
\newcommand{\Suggestion}[2]{%
	\ensuremath{%
		\begin{array}{l}%
			\SuggestLine{#1}{#2}{1} \\
			\SuggestLine{#1}{#2}{2} \\
			\SuggestLine{#1}{#2}{3} \\
			\SuggestLine{#1}{#2}{4}
		\end{array}%
	}%
}

\newcommand{\SuggestLine}[3]{%
			\ifthenelse{\equal{#1}{#3}}{}{\gamma_{#3} - \GetCell{#3}{#2}{dettmpb}\alg{dettmpb} \cdot \gamma_{#1}}%
}

% Получение значения по номеру
\newcommand{\GetCell}[3]{%
	\ifcase #1 \of %
		\or\ifcase #2 \of \or\Move{#3}{detaa} \or\Move{#3}{detab} \or\Move{#3}{detac} \or\Move{#3}{detad} \fi%
		\or\ifcase #2 \of \or\Move{#3}{detba} \or\Move{#3}{detbb} \or\Move{#3}{detbc} \or\Move{#3}{detbd} \fi%
		\or\ifcase #2 \of \or\Move{#3}{detca} \or\Move{#3}{detcb} \or\Move{#3}{detcc} \or\Move{#3}{detcd} \fi%
		\or\ifcase #2 \of \or\Move{#3}{detda} \or\Move{#3}{detdb} \or\Move{#3}{detdc} \or\Move{#3}{detdd} \fi%
	\fi%
}

% Получение нулей в 4х4
\newcommand{\ModifyDet}[2]{%	
	\ToNull{tmpb}	
	\whiledo{\value{tmpb}<4}{%
		\Inc{tmpb} \GetCell{\arabic{tmpb}}{#2}{dettmpc} \SubRow{\arabic{tmpb}}{#1}{dettmpc}%
	}%
}

% Вычитание строк
\newcommand{\SubRow}[3]{%
	\ifcase #1 \of %
		\or\ifcase #2 \of %
			\or%
			\or \setcounter{detaa}{\value{detaa}-\value{detba}*\value{#3}}%
				\setcounter{detab}{\value{detab}-\value{detbb}*\value{#3}}%
				\setcounter{detac}{\value{detac}-\value{detbc}*\value{#3}}%
				\setcounter{detad}{\value{detad}-\value{detbd}*\value{#3}}%
			\or \setcounter{detaa}{\value{detaa}-\value{detca}*\value{#3}}%
				\setcounter{detab}{\value{detab}-\value{detcb}*\value{#3}}%
				\setcounter{detac}{\value{detac}-\value{detcc}*\value{#3}}%
				\setcounter{detad}{\value{detad}-\value{detcd}*\value{#3}}%
			\or \setcounter{detaa}{\value{detaa}-\value{detda}*\value{#3}}%
				\setcounter{detab}{\value{detab}-\value{detdb}*\value{#3}}%
				\setcounter{detac}{\value{detac}-\value{detdc}*\value{#3}}%
				\setcounter{detad}{\value{detad}-\value{detdd}*\value{#3}}%
			\fi%
		\or\ifcase #2 \of %
			\or \setcounter{detba}{\value{detba}-\value{detaa}*\value{#3}}%
				\setcounter{detbb}{\value{detbb}-\value{detab}*\value{#3}}%
				\setcounter{detbc}{\value{detbc}-\value{detac}*\value{#3}}%
				\setcounter{detbd}{\value{detbd}-\value{detad}*\value{#3}}%
			\or %
			\or \setcounter{detba}{\value{detba}-\value{detca}*\value{#3}}%
				\setcounter{detbb}{\value{detbb}-\value{detcb}*\value{#3}}%
				\setcounter{detbc}{\value{detbc}-\value{detcc}*\value{#3}}%
				\setcounter{detbd}{\value{detbd}-\value{detcd}*\value{#3}}%
			\or \setcounter{detba}{\value{detba}-\value{detda}*\value{#3}}%
				\setcounter{detbb}{\value{detbb}-\value{detdb}*\value{#3}}%
				\setcounter{detbc}{\value{detbc}-\value{detdc}*\value{#3}}%
				\setcounter{detbd}{\value{detbd}-\value{detdd}*\value{#3}}%
			\fi%
		\or\ifcase #2 \of %
			\or \setcounter{detca}{\value{detca}-\value{detaa}*\value{#3}}%
				\setcounter{detcb}{\value{detcb}-\value{detab}*\value{#3}}%
				\setcounter{detcc}{\value{detcc}-\value{detac}*\value{#3}}%
				\setcounter{detcd}{\value{detcd}-\value{detad}*\value{#3}}%
			\or \setcounter{detca}{\value{detca}-\value{detba}*\value{#3}}%
				\setcounter{detcb}{\value{detcb}-\value{detbb}*\value{#3}}%
				\setcounter{detcc}{\value{detcc}-\value{detbc}*\value{#3}}%
				\setcounter{detcd}{\value{detcd}-\value{detbd}*\value{#3}}%
			\or %
			\or \setcounter{detca}{\value{detca}-\value{detda}*\value{#3}}%
				\setcounter{detcb}{\value{detcb}-\value{detdb}*\value{#3}}%
				\setcounter{detcc}{\value{detcc}-\value{detdc}*\value{#3}}%
				\setcounter{detcd}{\value{detcd}-\value{detdd}*\value{#3}}%
			\fi%
		\or\ifcase #2 \of %
			\or \setcounter{detda}{\value{detda}-\value{detaa}*\value{#3}}%
				\setcounter{detdb}{\value{detdb}-\value{detab}*\value{#3}}%
				\setcounter{detdc}{\value{detdc}-\value{detac}*\value{#3}}%
				\setcounter{detdd}{\value{detdd}-\value{detad}*\value{#3}}%
			\or \setcounter{detda}{\value{detda}-\value{detba}*\value{#3}}%
				\setcounter{detdb}{\value{detdb}-\value{detbb}*\value{#3}}%
				\setcounter{detdc}{\value{detdc}-\value{detbc}*\value{#3}}%
				\setcounter{detdd}{\value{detdd}-\value{detbd}*\value{#3}}%
			\or \setcounter{detda}{\value{detda}-\value{detca}*\value{#3}}%
				\setcounter{detdb}{\value{detdb}-\value{detcb}*\value{#3}}%
				\setcounter{detdc}{\value{detdc}-\value{detcc}*\value{#3}}%
				\setcounter{detdd}{\value{detdd}-\value{detcd}*\value{#3}}%
			\or %
			\fi%
	\fi%
}

% Переход от 4х4 к 3х3
\newcommand{\ToDefault}[2]{%
	\ifcase #2 \of %
		\or \Move{temp}{detaa}\Move{detaa}{detab}\Move{detab}{detac}\Move{detac}{detad}\Move{detad}{temp} %
			\Move{temp}{detba}\Move{detba}{detbb}\Move{detbb}{detbc}\Move{detbc}{detbd}\Move{detbd}{temp} %
			\Move{temp}{detca}\Move{detca}{detcb}\Move{detcb}{detcc}\Move{detcc}{detcd}\Move{detcd}{temp} %
			\Move{temp}{detda}\Move{detda}{detdb}\Move{detdb}{detdc}\Move{detdc}{detdd}\Move{detdd}{temp} %
		\or \Move{temp}{detab}\Move{detab}{detac}\Move{detac}{detad}\Move{detad}{temp} %
			\Move{temp}{detbb}\Move{detbb}{detbc}\Move{detbc}{detbd}\Move{detbd}{temp} %
			\Move{temp}{detcb}\Move{detcb}{detcc}\Move{detcc}{detcd}\Move{detcd}{temp} %
			\Move{temp}{detdb}\Move{detdb}{detdc}\Move{detdc}{detdd}\Move{detdd}{temp} %
		\or \Move{temp}{detac}\Move{detac}{detad}\Move{detad}{temp} %
			\Move{temp}{detbc}\Move{detbc}{detbd}\Move{detbd}{temp} %
			\Move{temp}{detcc}\Move{detcc}{detcd}\Move{detcd}{temp} %
			\Move{temp}{detdc}\Move{detdc}{detdd}\Move{detdd}{temp} %
		\or \Move{temp}{detad}\Move{detad}{temp} %
			\Move{temp}{detbd}\Move{detbd}{temp} %
			\Move{temp}{detcd}\Move{detcd}{temp} %
			\Move{temp}{detdd}\Move{detdd}{temp} %
	\fi%
	\ifcase #1 \of %
		\or \Move{temp}{detaa}\Move{detaa}{detba}\Move{detba}{detca}\Move{detca}{detda}\Move{detda}{temp} %
			\Move{temp}{detab}\Move{detab}{detbb}\Move{detbb}{detcb}\Move{detcb}{detdb}\Move{detdb}{temp} %
			\Move{temp}{detac}\Move{detac}{detbc}\Move{detbc}{detcc}\Move{detcc}{detdc}\Move{detdc}{temp} %
			\Move{temp}{detad}\Move{detad}{detbd}\Move{detbd}{detcd}\Move{detcd}{detdd}\Move{detdd}{temp} %
		\or \Move{temp}{detba}\Move{detba}{detca}\Move{detca}{detda}\Move{detda}{temp} %
			\Move{temp}{detbb}\Move{detbb}{detcb}\Move{detcb}{detdb}\Move{detdb}{temp} %
			\Move{temp}{detbc}\Move{detbc}{detcc}\Move{detcc}{detdc}\Move{detdc}{temp} %
			\Move{temp}{detbd}\Move{detbd}{detcd}\Move{detcd}{detdd}\Move{detdd}{temp} %
		\or \Move{temp}{detca}\Move{detca}{detda}\Move{detda}{temp} %
			\Move{temp}{detcb}\Move{detcb}{detdb}\Move{detdb}{temp} %
			\Move{temp}{detcc}\Move{detcc}{detdc}\Move{detdc}{temp} %
			\Move{temp}{detcd}\Move{detcd}{detdd}\Move{detdd}{temp} %
		\or \Move{temp}{detda}\Move{detda}{temp} %
			\Move{temp}{detdb}\Move{detdb}{temp} %
			\Move{temp}{detdc}\Move{detdc}{temp} %
			\Move{temp}{detdd}\Move{detdd}{temp} %
	\fi%
} 

%	Решение ситемы 3x3 методом Крамера
\newcommand\Kramer{%
	\ensuremath{%
		\D\begin{cases} %
			\ToNull{detad}%
			\ToNull{detbd}%
			\ToNull{detcd}%
			\Multi{dettmpa}{detaa}{detda}\Ipp{detad}{dettmpa}%
			\Multi{dettmpa}{detab}{detdb}\Ipp{detad}{dettmpa}%
			\Multi{dettmpa}{detac}{detdc}\Ipp{detad}{dettmpa}%
			\SmartPrintValue[0]{detaa} \IfNull{detaa}{}{x} \SmartPrintValue{detab} \IfNull{detab}{}{y} \SmartPrintValue{detac} \IfNull{detac}{}{z} = \arabic{detad} \\%
			\Multi{dettmpa}{detba}{detda}\Ipp{detbd}{dettmpa}%
			\Multi{dettmpa}{detbb}{detdb}\Ipp{detbd}{dettmpa}%
			\Multi{dettmpa}{detbc}{detdc}\Ipp{detbd}{dettmpa}%
			\SmartPrintValue[0]{detba} \IfNull{detba}{}{x} \SmartPrintValue{detbb} \IfNull{detbb}{}{y} \SmartPrintValue{detbc} \IfNull{detbc}{}{z} = \arabic{detbd} \\%
			\Multi{dettmpa}{detca}{detda}\Ipp{detcd}{dettmpa}%
			\Multi{dettmpa}{detcb}{detdb}\Ipp{detcd}{dettmpa}%
			\Multi{dettmpa}{detcc}{detdc}\Ipp{detcd}{dettmpa}%
			\SmartPrintValue[0]{detca} \IfNull{detca}{}{x} \SmartPrintValue{detcb} \IfNull{detcb}{}{y} \SmartPrintValue{detcc} \IfNull{detcc}{}{z} = \arabic{detcd} \\%
		\end{cases}%
	}%
}

% Вывод заданного количества систем 3х3. Можно указать число, для генерации конкретных вариантов.
\newcommand{\MultiKramer}[2][\time]{%
	\reinitrand[seed=#1, first=-9, last=9]%
	\ToNull{tmpa}%
	\whiledo{\value{tmpa}<#2}{%
		\Inc{tmpa}\Example{\InitDet[4]\Kramer}{}%
%		\ifthenelse{\value{tmpa}<#2}{\Inc{tmpa}\Inc{primcount}\InitDet[4]\theprimcount\ \Kramer}{} \BigPar%
	 }%
}

% Решение системы 3х3 методом Крамера
\newcommand\SolveKramer[1]{%
	\ensuremath{%
		\Kramer \Longrightarrow%
	}
	\ifthenelse{\equal{#1}{S}}{%
		\ensuremath{%	
			\Delta = \SolveDetThree{O} \Move{detdd}{dettmpb}%
		}%
		\ensuremath{%
			\Swap{detaa}{detad}\Swap{detba}{detbd}\Swap{detca}{detcd} %
			\Delta_x = \DetThree = \Multi{dettmpb}{detda}{detdd}\arabic{dettmpb} \qquad %
			\Swap{detad}{detaa}\Swap{detbd}{detba}\Swap{detcd}{detca} %
			\Swap{detab}{detad}\Swap{detbb}{detbd}\Swap{detcb}{detcd} %
			\Delta_y = \DetThree = \Multi{dettmpb}{detdb}{detdd}\arabic{dettmpb}%
		}%
		\ensuremath{%
			\Swap{detad}{detab}\Swap{detbd}{detbb}\Swap{detcd}{detcb} %
			\Swap{detac}{detad}\Swap{detbc}{detbd}\Swap{detcc}{detcd} %
			\Delta_z = \DetThree = \Multi{dettmpb}{detdc}{detdd}\arabic{dettmpb} \qquad %
			\Swap{detad}{detac}\Swap{detbd}{detbc}\Swap{detcd}{detcc} %
			\begin{cases} %
				x = \Delta_x / \Delta = \arabic{detda} \\ %
				y = \Delta_y / \Delta = \arabic{detdb} \\ %
				z = \Delta_z / \Delta = \arabic{detdc} %
			\end{cases} %
		}%
	}{%
		\ensuremath{%
			\begin{cases} %
				x = \arabic{detda} \\ %
				y = \arabic{detdb} \\ %
				z = \arabic{detdc} %
			\end{cases} %
		}%
	}%
}

% Много решений методом Крамера
\newcommand{\MultiSolveKramer}[3][\time]{%
	\reinitrand[seed=#1, first=-9, last=9]%
	\ToNull{tmpa}%
	\whiledo{\value{tmpa}<#2}{%
		\Inc{tmpa}\Example{\InitDet[4]\SolveKramer{#3}}{}%
	 }%
}

% Печать матрицы
\newcommand\PrintMatrix[1]{%
	\ifcase #1 \or \or \ensuremath{%
		\begin{array}{rr}%
			\arabic{detaa} & \arabic{detab} \\ %
			\arabic{detba} & \arabic{detbb}%
		\end{array}%
	}%
	\or \ensuremath{%
		\begin{array}{rrr}%
			\arabic{detaa} & \arabic{detab} & \arabic{detac}\\ %
			\arabic{detba} & \arabic{detbb} & \arabic{detbc}\\ %
			\arabic{detca} & \arabic{detcb} & \arabic{detcc}%
		\end{array}%
	}%
	\or \ensuremath{%
		\begin{array}{rrrr}%
			\arabic{detaa} & \arabic{detab} & \arabic{detac} & \arabic{detad}\\ %
			\arabic{detba} & \arabic{detbb} & \arabic{detbc} & \arabic{detbd}\\ %
			\arabic{detca} & \arabic{detcb} & \arabic{detcc} & \arabic{detcd}\\ %
			\arabic{detda} & \arabic{detdb} & \arabic{detdc} & \arabic{detdd} %
		\end{array}%
	}%
	\or \ensuremath{%
		\begin{array}{rrrrr}%
			\arabic{detaa} & \arabic{detab} & \arabic{detac} & \arabic{detad} & \arabic{detae}\\ %
			\arabic{detba} & \arabic{detbb} & \arabic{detbc} & \arabic{detbd} & \arabic{detbe}\\ %
			\arabic{detca} & \arabic{detcb} & \arabic{detcc} & \arabic{detcd} & \arabic{detce}\\ %
			\arabic{detda} & \arabic{detdb} & \arabic{detdc} & \arabic{detdd} & \arabic{detde}\\ %
			\arabic{detea} & \arabic{deteb} & \arabic{detec} & \arabic{deted} & \arabic{detee} %
		\end{array}%
	}%
	\fi
}


% Генерация обратной матрицы
\newcommand\InvertMatrix[1]{
	\ensuremath{\left(\PrintMatrix{#1}\right)^{-1}}%
}

% Много задач на обратную матрицу
\newcommand{\MultiInvertMatrix}[3][\time]{%
	\reinitrand[seed=#1, first=-9, last=9]%
	\ToNull{tmpa}%
	\whiledo{\value{tmpa}<#2}{%
		\Inc{tmpa}\Example{\InitDet[#3]\InvertMatrix{#3}.}{}%
%		\ifthenelse{\value{tmpa}<#2}{\Inc{tmpa}\Inc{primcount}\InitDet\theprimcount\ \InvertMatrix{#3}.}{} \BigPar%
	 }%
}

% Транспонирование матрицы
\newcommand\Transpose[1][3]{%
	\Swap{detab}{detba}\Swap{detac}{detca}\Swap{detbc}{detcb}%
	\ifthenelse{\equal{#1}{4}}{\Swap{detad}{detda}\Swap{detbd}{detdb}\Swap{detcd}{detdc}}{}%
}

% Нахождение обратной матрицы с помощью алгебраических дополнений
\newcommand\SolveInvertMatrix[2]{%
	\ensuremath{\InvertMatrix{#1}} = \ensuremath{\ifthenelse{\equal{#2}{S}}{\frac{\D 1}{\D \det}\times}{}}%
	\ifcase #1 \or \or \ensuremath{%
		\setcounter{dettmpb}{\value{detaa}*\value{detbb} - \value{detab}*\value{detba}}
		\Swap{detaa}{detbb}\Swap{detab}{detba}\Inverse{detab}\Inverse{detba}%
		\ifthenelse{\equal{#2}{S}}{\left(\PrintMatrix{2}\right)^{T} = }{}\Transpose %
		\IfNegate{dettmpb}{-\Inverse{dettmpb}}{}%
		\IfOne{dettmpb}{}{%
			\D\frac{1}{\arabic{dettmpb}}\times%
		}\left(\PrintMatrix{2}\right).
	} \or \small{\ensuremath{%
		\vphantom{\SolveDetThree{O}}\ifthenelse{\equal{#2}{S}}{
			\begin{pmatrix}%
			 \left| \begin{array}{rr} \arabic{detbb} & \arabic{detbc} \\ \arabic{detcb} & \arabic{detcc} \end{array}\right| &% 
			-\left| \begin{array}{rr} \arabic{detba} & \arabic{detbc} \\ \arabic{detca} & \arabic{detcc} \end{array}\right| &% 
			 \left| \begin{array}{rr} \arabic{detba} & \arabic{detbb} \\ \arabic{detca} & \arabic{detcb} \end{array}\right| \\% 
			-\left| \begin{array}{rr} \arabic{detab} & \arabic{detac} \\ \arabic{detcb} & \arabic{detcc} \end{array}\right| &% 
			 \left| \begin{array}{rr} \arabic{detaa} & \arabic{detac} \\ \arabic{detca} & \arabic{detcc} \end{array}\right| &% 
			-\left| \begin{array}{rr} \arabic{detaa} & \arabic{detab} \\ \arabic{detca} & \arabic{detcb} \end{array}\right| \\% 
			 \left| \begin{array}{rr} \arabic{detab} & \arabic{detac} \\ \arabic{detbb} & \arabic{detbc} \end{array}\right| &% 
			-\left| \begin{array}{rr} \arabic{detaa} & \arabic{detac} \\ \arabic{detba} & \arabic{detbc} \end{array}\right| &% 
			 \left| \begin{array}{rr} \arabic{detaa} & \arabic{detab} \\ \arabic{detba} & \arabic{detbb} \end{array}\right| % 
		\end{pmatrix}^{T} = }{}%
		\setcounter{mataaa}{ \value{detbb}*\value{detcc} - \value{detbc}*\value{detcb}}%
		\setcounter{mataab}{-\value{detba}*\value{detcc} + \value{detbc}*\value{detca}}%
		\setcounter{mataac}{ \value{detba}*\value{detcb} - \value{detbb}*\value{detca}}%
		\setcounter{mataba}{-\value{detab}*\value{detcc} + \value{detac}*\value{detcb}}%
		\setcounter{matabb}{ \value{detaa}*\value{detcc} - \value{detac}*\value{detca}}%
		\setcounter{matabc}{-\value{detaa}*\value{detcb} + \value{detab}*\value{detca}}%
		\setcounter{mataca}{ \value{detab}*\value{detbc} - \value{detac}*\value{detbb}}%
		\setcounter{matacb}{-\value{detaa}*\value{detbc} + \value{detac}*\value{detba}}%
		\setcounter{matacc}{ \value{detaa}*\value{detbb} - \value{detab}*\value{detba}}%
		\Move{detaa}{mataaa}\Move{detab}{mataab}\Move{detac}{mataac}%
		\Move{detba}{mataba}\Move{detbb}{matabb}\Move{detbc}{matabc}%
		\Move{detca}{mataca}\Move{detcb}{matacb}\Move{detcc}{matacc}%
		\ifthenelse{\equal{#2}{S}}{\frac{\D 1}{\D \arabic{dettmpb}}\times\left(\PrintMatrix{3}\right)^{T} =}{}%
		\Transpose%
		\IfNegate{dettmpb}{-\Inverse{dettmpb}}{}%
		\IfOne{dettmpb}{}{%
			\D\frac{1}{\arabic{dettmpb}}\times%
		}\left(\PrintMatrix{3}\right).
	}}\fi%
}

% Много решений обратной матрицы
\newcommand\MultiSolveInvertMatrix[4][\time]{%
	\reinitrand[seed=#1, first=-9, last=9]%
	\ToNull{tmpa}%
	\whiledo{\value{tmpa}<#2}{%
		\Inc{tmpa}\Example{\InitDet[#3]\SolveInvertMatrix{#3}{#4}}{}%
	 }%
}

% Система с обратной матрицей
\newcommand\InvertSystem{%
	\Kramer
}

% Много систем с обратной матрицей
\newcommand\MultiInvertSystem[2][\time]{%
	\reinitrand[seed=#1, first=-9, last=9]%
	\ToNull{tmpa}%
	\whiledo{\value{tmpa}<#2}{%
		\Inc{tmpa}\Example{\InitDet[4]\InvertSystem}{}%
%		\ifthenelse{\value{tmpa}<#2}{\Inc{tmpa}\Inc{primcount}\InitDet[4]\theprimcount\ \InvertSystem}{} \BigPar%
	 }%
}

% Решение системы с обратной матрицей
\newcommand\SolveInvertSystem[1]{%
	\ifthenelse{\equal{#1}{O}}{\SolveKramer{#1}}{%
		%
	}%
}

% Много решений систем с обратной матрицей
\newcommand\MultiSolveInvertSystem[3][\time]{%
	\reinitrand[seed=#1, first=-9, last=9]%
	\ToNull{tmpa}%
	\whiledo{\value{tmpa}<#2}{%
		\Inc{tmpa}\Example{\InitDet[4]\SolveInvertSystem{#3}}{}%
	 }%
}

\newcommand\Vyp{\centerline{Выпуклый.}}

\newcommand\Vog{\centerline{Вогнутый.}}

\newcommand\Inside{\centerline{Внутри.}}

\newcommand\Outside{\centerline{Вне.}}

\newcommand\Tup{\centerline{В тупом.}}

\newcommand\Ost{\centerline{В остром.}}

\newcommand\Oang{\centerline{Общий угол.}}

\newcommand\Sang{\centerline{Смежные углы.}}

\newcommand\Vang{\centerline{Вертикальные углы.}}

\newcommand\Ellipse{\centerline{Эллипс.}}

\newcommand\Hyperbola{\centerline{Гипербола.}}

\newcommand\Parabola{\centerline{Парабола.}}


\newcommand{\Theme}{
	\textbf{%
		\ifcase \arabic{fifteencount} %
			\or Алгебраические операции над матрицами%
			\or Определители%
			\or Обратные матрицы%
			\or Квадратные системы линейных алгебраических уравнений%
			\or Прямоугольные системы линейных алгебраических уравнений %
			\or Векторная алгебра%
			\or Прямые на плоскости%
			\or Плоскости в пространстве%
			\or Прямые в пространстве%
			\or Собственные числа и собственные векторы%
			\or Комплексные числа
			\or Базис пространства
		\fi%
	}
}

\input{tasknumber.tex}
\input{name.tex}

\Inc{taskcount}
\Inc{tmpsmart}
\hypersetup{%
	unicode,%
	colorlinks=true,%
%	pdfpagemode=FullScreen,%
%	pdfpagetransition=Dissolve,%
	pdftitle={Домашняя работа \arabic{fifteencount} по предмету "Линейная алгебра и геометрия"},%
	pdfauthor={Ali Shadow, Alira},%
	pdfsubject={Домашнее задание},%
	pdfkeywords={Линейная алгебра, Домашняя работа}%
}


\title{Методическое пособие по теме <<Матрицы>> \\ из курса <<Линейная алгебра и аналитическая геометрия>>}
\author{Денисова И. П. \and Егорова Е. К. \and Костиков Ю. А. \and Мокряков А. В.}
\date{\today}
